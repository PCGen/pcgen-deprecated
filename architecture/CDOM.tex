\documentclass[12pt,letterpaper]{article}

\usepackage{graphicx,wrapfig,colordvi,color,citesort,times,array,subfigure,floatflt,makeidx,framed,boxedminipage,url,hyperref}

\graphicspath{{images/}}
\definecolor{UglyBrown}{rgb}{0.7,0.4,0.1} 
\newcommand{\newtext}[1]{\textcolor{UglyBrown}{#1}}

\newcommand{\contributors}{
Thomas Parker\\
Devon Jones\\
James Dempsey\\
Aaron Divinsky\\
Tir Gwaith (Andrew McDougall)\\
Frank Kliewe\\
Eddy Anthony\\
Thomas Clegg\\
Paul W. King\\
Paul Grosse\\
Terry FitzSimons\\
Chris Chandler\\
Martijn Verburg\\
}
\newcommand{\lastupdate}{June 29, 2007}
\newcommand{\versionEOS}{0.51}
\newcommand{\lastversionEOS}{0.4}

\makeindex

\hoffset=-0.6in
\voffset=-0.6in
\textwidth=6.5in
\textheight=9.0in
\parindent=0.25in
\parskip=1.0ex

\newcommand{\version}{\versionEOS{} }
\newcommand{\lastversion}{\lastversionEOS{} }

\definecolor{Red}{rgb}{1,0,0} 
\definecolor{Blue}{rgb}{0,0,1} 
\definecolor{Black}{rgb}{0,0,0} 
\newcommand{\redtext}[1]{\textcolor{Red}{#1}}
\newcommand{\revisedtext}[1]{\textcolor{Blue}{#1}}

\newcommand{\newtextPointZeroTwo}[1]{#1}
\newcommand{\revisedtextPointZeroTwo}[1]{#1}

\newcommand{\newtextPointZeroThree}[1]{#1}
\newcommand{\revisedtextPointZeroThree}[1]{#1}

\newcommand{\newtextPointZeroFour}[1]{#1}
\newcommand{\revisedtextPointZeroFour}[1]{#1}
\newcommand{\deletedTextPointZeroFour}[1]{}

\newcommand{\newtextPointZeroFive}[1]{#1}
\newcommand{\revisedtextPointZeroFive}[1]{#1}
\newcommand{\deletedTextPointZeroFive}[1]{}

\newcommand{\newtextPointTwo}[1]{#1}
\newcommand{\revisedtextPointTwo}[1]{#1}
\newcommand{\deletedTextPointTwo}[1]{}

\newcommand{\newtextPointThree}[1]{#1}
\newcommand{\revisedtextPointThree}[1]{#1}
\newcommand{\deletedTextPointThree}[1]{}

\newcommand{\newtextPointFour}[1]{#1}
\newcommand{\revisedtextPointFour}[1]{#1}
\newcommand{\deletedTextPointFour}[1]{}

\newcommand{\newtextPointFive}[1]{\newtext{#1}}
\newcommand{\revisedtextPointFive}[1]{\revisedtext{#1}}
\newcommand{\deletedTextPointFive}[1]{}

\newcommand{\textem}[1]{\emph{#1}}
\newcommand{\ix}[1]{#1\index{#1}}
\newcommand{\tool}[1]{\textbf{\ix{#1}}}

\newcommand{\rejected}[1]{}

\newcommand{\nsection}[1]{\newpage \section{#1}}
\newcommand{\lsection}[1]{\section{#1}\label{#1}}
\newcommand{\lnsection}[1]{\nsection{#1}\label{#1}}
\newcommand{\lsubsection}[1]{\subsection{#1}\label{#1}}
\newcommand{\lnsubsection}[1]{\newpage\subsection{#1}\label{#1}}
\newcommand{\lsubsubsection}[1]{\subsubsection{#1}\label{#1}}
\newcommand{\lisection}[1]{\section{#1}\label{#1}\index{#1}}
\newcommand{\lisubsection}[1]{\subsection{#1}\label{#1}\index{#1}}
\newcommand{\lisubsubsection}[1]{\subsubsection{#1}\label{#1}\index{#1}}

\newcommand{\sidebar}[5]{
\noindent
\definecolor{shadecolor}{#2}{#3}
\begin{minipage}{\textwidth}
\setlength{\parskip}{-11pt}
\begin{shaded}\textbf{#1: #4}\end{shaded}
\begin{boxedminipage}{\textwidth}
#5
\end{boxedminipage}
\end{minipage}
}

\newcommand{\authors}{Members of the PCGen Development Team and Community (see contributors)}

\newcommand{\csidebar}[6]{
\noindent
\definecolor{shadecolor}{#2}{#3}
\begin{minipage}{\textwidth}
\setlength{\parskip}{-11pt}
\begin{shaded}\textbf{\textcolor{#6}{#1: #4}}\end{shaded}
\begin{boxedminipage}{\textwidth}
#5
\end{boxedminipage}
\end{minipage}
}

\newcommand{\sbdevon}[2]{\sidebar{Devon's Thoughts}{gray}{0.9}{#1}{#2}}
\newcommand{\sbjames}[2]{\sidebar{James' Thoughts}{gray}{0.9}{#1}{#2}}
\newcommand{\sbtom}[2]{\sidebar{Tom's Thoughts}{gray}{0.9}{#1}{#2}}

\newcommand{\sbnote}[2]{\sidebar{Note}{gray}{0.9}{#1}{#2}}
\newcommand{\sbwhatis}[2]{\sidebar{What Is}{gray}{0.8}{#1}{#2}}
\newcommand{\sbeffect}[2]{\sidebar{Effect}{gray}{0.8}{#1}{#2}}
\newcommand{\sbexception}[2]{\sidebar{Exception}{gray}{0.7}{#1}{#2}}
\newcommand{\sbwarning}[2]{\csidebar{Warning}{gray}{0.3}{#1}{#2}{white}}
\newcommand{\sbold}[2]{\csidebar{Original Proposal}{gray}{0.3}{#1}{#2}{white}}

\newcommand{\vbar}{\hspace{0.2mm}\rule{0.2mm}{3mm}\hspace{0.2mm}}

\newcommand{\openfig}{\begin{figure}[hbt]}
\newcommand{\closefig}[1]{\vspace*{-0.15in}\caption{#1}\end{figure}}

\newcommand{\openffig}[2]{\begin{floatingfigure}[#1]{#2}}
\newcommand{\closeffig}[1]{\caption{#1}\end{floatingfigure}}
\newcommand{\nocapcloseffig}{\end{floatingfigure}}

\newcommand{\mapffig}[4]{\openffig{#1}{#2}\includegraphics[scale=0.675]{#3}\closeffig{#4}}
\newcommand{\mapffigl}[3]{\mapffig{l}{#1}{#2}{#3}}
\newcommand{\mapffigr}[3]{\mapffig{r}{#1}{#2}{#3}}

%\newcommand{\mapfig}[2]{\openfig\centerline{\includegraphics[scale=0.675]{#1}}\closefig{#2}}
%\newcommand{\subf}[3]{\subfigure[#1]{#2\includegraphics[scale=0.675]{#3}}}
%\newcommand{\maptwofig}[7]{\openfig\begin{center}\mbox{\subf{#1}{#2}{#3}\hspace{0.25in}\subf{#4}{#5}{#6}}\end{center}\caption{#7}\end{figure}}
%\newcommand{\mapfigl}[3]{\mapfig{#2}{#3}}%\mapfig{l}{#1}{#2}{#3}}
%\newcommand{\mapfigr}[3]{\mapfig{#2}{#3}}%\mapfig{r}{#1}{#2}{#3}}
%\newcommand{\mapscalefig}[5]{\openffig{#1}{#2}\includegraphics[#3]{#4}\nocapcloseffig}

\newcommand{\mapscalefig}[3]{\openfig\centerline{\includegraphics[#1]{#2}}\closefig{#3}}
\newcommand{\mapfig}[2]{\openfig\centerline{\includegraphics[scale=0.675]{#1}}\closefig{#2}}
\newcommand{\subf}[3]{\subfigure[#1]{#2\includegraphics[scale=0.675]{#3}}}
\newcommand{\maptwofig}[7]{\openfig\begin{center}\mbox{\subf{#1}{#2}{#3}\hspace{0.25in}\subf{#4}{#5}{#6}}\end{center}\caption{#7}\end{figure}}

\renewcommand{\baselinestretch}{1.0}
\hyphenpenalty=9999
\exhyphenpenalty=9999
\pagestyle{empty}
\sloppy
%\thispagestyle{empty}


%============================================================================

\begin{document}

\begin{center}

\vspace*{2.5in}
{ \Huge PCGen 6.0 (``CDOM'') Architecture }

{ \huge Draft Proposal \versionEOS }

{ \large \authors }

\lastupdate

\end{center}

\vspace*{3in}

\noindent Portions \copyright  2006, 2007 Individually by \authors.
This work is licensed under the Creative Commons Attribution-NonCommercial-ShareAlike License.
To view a copy of this license, visit \url{http://creativecommons.org/licenses/by-nc-sa/2.5/}
or send a letter to Creative Commons, 543 Howard Street, 5th Floor,
San Francisco, California, 94105, USA.

Sun, Sun Microsystems, and Java, are trademarks or registered trademarks
of Sun Microsystems, Inc. in the United States and other countries.
All other trademarks mentioned are the property of their respective owners.

\newpage
\pagestyle{plain}
\setcounter{page}{1}
\pagenumbering{roman}
\tableofcontents

\listoffigures

\lnsection{PCGen Background}
\setcounter{page}{1}
\pagenumbering{arabic}

Over the course of the last few years, there has been a pending discussion of reworking
the core of the PCGen code base.  Today, PCGen is moving toward a release of version 5.12.
The current plan \revisedtextPointFour{to release version 5.12, followed by a release 5.14 
(used to do final cleanup before CDOM), which will then be followed by release 6.0.}
\deletedTextPointFour{to release version 6.0 as the next production release after 5.12.}
In order to prepare for a move to Character Document Object Model (``CDOM''), some
significant refactoring of the core code base has been suggested or is in process. 
It would be beneficial to avoid making changes in the refactoring that would 
negatively impact a transition to CDOM.  Thus, the plan for version 6.0 requires
defining the architecture in sufficient detail to make appropriate decisions about
changes to the core objects.

This document is an attempt to flesh out some specific architectural details of CDOM 
in order to assist in driving CDOM to implementation in the 5.13 \revisedtextPointFour{and 5.15}
alpha cycle\revisedtextPointFour{s} (in preparation for PCGen 6.0).

\lsection{Contributors}

\newtextPointZeroThree{
Contributors to this document, either directly or by feedback and comments:\\
\contributors
}

\lsection{Background of the PCGen code base}

Currently, the PCGen code base contains seven major packages (or package groups)
that combine to form PCGen. 
These packages are:
\begin{itemize}
\item pcgen.core.*
\item pcgen.gui.*
\item pcgen.io.*
\item pcgen.persistence.*\index{Persistence}
\item pcgen.util.*
\item plugin.*\index{plugins}
\item gmgen.*
\end{itemize}

The PCGen code base is, by some standards, a large code base.  
\revisedtextPointZeroFive{The production code is composed of
1691 Class files (with more than 3,000 Classes counting inner Classes).  
The largest Class being upwards of 10,000 lines (including comments).  
PCGen also has more than 100,000 of executable lines of code.
}
This is clearly a significant code base, and is not easily modified.

Currently, the PCGen core components generally extend one Class, called PObject.  This Class
provides base functionality that is supposed to be common to the Classes used to build a
Player Character.  Each PObject is aware of the PObjects that it contains, and maintains
those associations in (often typesafe) lists.  This CDOM proposal would replace these
lists with a ``tree-like'' architecture where the `contains' association would be implicit
within the tree structure.  This would lighten the memory footprint (and complexity) of
PObject and its subclasses.

\lsection{Conventions in this Document}

\newtextPointZeroTwo{
The following conventions are used throughout this document, and should be respected when
editing this proposal.
}

\newtextPointZeroTwo{
Significant information that is new to a specific version of this document is highlighted in
\newtext{Ugly Brown}. Information that has been revised
due to feedback is highlighted in \revisedtext{Blue}.  
This eases identification of changes from a previous revision. 
Spelling corrections and small wording clarifications are not highlighted.
\revisedtext{Currently highlighted are changes from \lastversion to \versionEOS .}
}

\newtextPointZeroTwo{
Please note: At times, color will not be maintained over a page break, so that if the page
breaks in the middle of a colored paragraph, the last part of the text will
incorrectly be printed in black.
\newtextPointZeroFive{There is also a bug that can accidentally start a new page in color,
when the start of the paragraph was black.  The moral is: be wary of color changes over page
breaks.\footnote{\ldots and don't use a Windows \LaTeX{} client if you're using colors :/}
}}

\newtextPointZeroTwo{
Packages, Classes, method calls, and the null value are shown in a \texttt{typewriter} font.
}

\newtextPointThree{
Occasionally, but not in code examples, method will be followed by a
null parameter list () even if there actually
parameters in the Java method call. This is for reading ease of this document and should
not be taken as a literal interpretation of the code.
}

\redtext{
***PLEASE NOTE***: If you are providing feedback on this document, it is built in \LaTeX, and 
therefore, it may undergo automatic section renumbering as text is added.  Therefore, please
make sure - at a minimum - to include the document revision and section number
that you are commenting on.  It is extremelly helpful for you to include the
section title along with the section number and document revision, as the section
numbers do not appear in the raw document, while the section titles do.
}

\lnsection{Scope of this Document}

\newtextPointZeroTwo{
This document is intended to provide an architecture for CDOM, not an entire architecture definition 
for PCGen.  In order to be explicit about the scope of the changes proposed here the following
observations are made:
\begin{itemize}
\item This proposal makes no suggestions about the structure, layout, or design of the 
PCGen \ix{GUI}, \revisedtextPointZeroThree{with the exception that:
\item This proposal will provide an interface which the \ix{GUI} should use to interface to the 
PCGen core (the CDOM), to display the current state of the PC
\item This proposal will provide an interface which \ix{output} processing and character \ix{save}
operations should be programmed against
}
\item This proposal will not change the scope of ownership or design of the \texttt{\ix{GMBus}}
producer/consumer interface
\item The only \ix{plugins} within the scope of this proposal are those within
\texttt{plugin.lsttokens.*}
\item The \ix{Prerequisite} architecture is out of the scope of this proposal.  The Prerequisite
Classes should remain as unchanged as possible (with some potential package name changes)
\item The \ix{export} system for output sheets is out of the scope of this proposal (except for required
incidental changes to support changes to the core)
\item The \ix{export} system for exporting back into an \ix{LST file} (from the editors) IS within
the scope of this proposal.
\item \revisedtextPointZeroFour{The \ix{loader} system and hierarchy is partially in
the scope of this proposal.}  Some
changes will be driven by changes to the tags and core, but the primary architecture
of the \ix{loader} system will remain unchanged.  \newtextPointZeroFour{However, there may
be some significant advantages to additional Loaders/Tokens\index{token} to handle complex Tokens
(ADD and CHOOSE), as well as some significant reduction
in the number of Classes required to perform the import of Tokens. 
These potential changes do not alter the general structure of the
\ix{loader} system, but refine the architecture to better handle the file structures in the
CDOM environment.}
\end{itemize}
}

\lnsection{Code Assumptions}

The following are proposed assumptions about the future PCGen CDOM code.  In some cases
(such as avoiding tangles) these are so-called `best practices' for coding, while in others,
they represent a subjective architectural decision that may have alternatives.  I hope this
is understood, but in early revisions of this document, all of these assumptions should be
considered open for discussion and alteration.  (Also note these are intended to be beyond
the coding guidelines presented on the PCGen Wiki--most of those are micro-guidelines
related to formatting, syntax, et. al.)

\lsubsection{Minimize Code Tangles}

Currently, the PCGen code base has a large presence of tangles.  These are circular references between
Classes.  This causes two significant issues.  First, tangles between objects often lead to changes
impacting other Classes, resulting in larger changes than would be necessary in a system without tangles. 
Second, tangles are a bane to good unit {testing}.  If Classes are in a tangle, they cannot be isolated. 
Performing independent \ix{unit test}s on the Classes is impossible.  In addition, tangles prevent quick
incremental compiles, because the dependency tree for a given Class can rapidly become very large. 
The largest tangle of Classes in PCGen is nearly 400 Classes, so a significant portion of the code
is intertwined in this fashion.

The PCGen CDOM architecture will attempt to minimize the number of code
tangles in order to facilitate Class independence and \ix{unit test}ing.  Stated another way,
the PCGen code base will work to become a directed acyclic graph.  

For information on why this is important, you can read the following
document:\footnote{Actually much of the series is interesting -� Tom Parker 11/18/06}
\url{http://www.objectmentor.com/resources/articles/granularity.pdf}

Note the use of a directed acyclic graph could be enforced using software.  For example,
\ix{SonarJ} is available free of charge (in 30 day doses) to open source
programmers.\footnote{I have no association with the company that produces SonarJ, I just
found it when I was looking for something to help enforce this behavior.  I have a
30-day open source programmer license and am starting to use it with PCGen �- Tom Parker 11/18/06}

\revisedtextPointZeroThree{
Nothing in minimizing code tangles should be taken as altering the publisher/subscriber
interface that exists in \texttt{\ix{GMBus}} (and is used to connect the \ix{GUI} to the core).  While this
is a tangle today, I don't think a pub/sub should naturally produce a
tangle.
}

\sbeffect{File Movement}{The
PCGen code base will need to be significantly restructured (some of it
simply moving files) in order to make the packages into a directed acyclic graph. 
\revisedtextPointZeroFive{
If done properly, SVN will preserve the history of the Classes despite
having packages renamed or Classes renamed.
Due to concurrent file splitting/refactoring, this may
muck up much of the ability to trace backwards changes in the code base. 
}
However, if done concurrently with CDOM, it's probably moot how much damage is done
to history tracing, since so much code will undergo a severe overhaul.  A proposal
for the new package structure is available in section \ref{Proposed Packages},
Proposed Packages.}

\sbexception{Listener/Event Combinations}{When
there is a Listener/Event combination, the cost of breaking a
tangle (a single use \texttt{Interface}) is considered
wasteful of memory and generally confusing.
Listener/Event tangles that contain two and ONLY two Classes are permitted.}

\lsubsection{Maintain Loose Association}

\revisedtextPointZeroFive{
The \ix{GUI} is currently too tightly coupled with the core and vice versa. 
One example of this can be found in the process of levelling up
(from the Summary tab and the Classes tab).
}
The PCGen \ix{GUI} should be loosely associated with the core PCGen package(s).

\sbeffect{Event Model}{\revisedtextPointZeroFive{A model to
communicate changes from the PCGen core to the
PCGen \ix{GUI} must be developed.  This may be events that are handled in a Producer/Consumer
model (the current working proposal) or by some other means.  In general, the association 
between the core and the \ix{GUI} must be a loose association and except for the actual values 
stored in the view of the \ix{GUI} (such as inside a \texttt{JTextField}), there should not
be an extra copy of the information from PlayerCharacter stored within the \ix{GUI}.
I don't know if this model will break the skins, because I haven't looked at how those work.
}}

\lsubsection{Contracts are prohibited}

\index{contract}
In no case, should the code, \textem{or a programmer},
be required to know that taking a specific action also requires another action to take
place.

\sbeffect{Reciprocal Awareness is a Challenge}{A ``tree'' or ``tree-like''
structure where there is reciprocal awareness by both
the parent and child is prohibited.  While it is possible to hide these contracts in a
wrapper object, doing so in a way that hides the interface for each sub-step while not
violating Assumption \ref{Minimize Code Tangles} is postulated
as being very difficult.  Note: The use of
parent-child only or child-parent only awareness is recognized to be a performance problem
(of forcing frequent linear-time `tree' searches), so an external Class must be aware of
all associations in order to guarantee good performance.}

\sbexception{Clone}{Proper maintenance of \texttt{\ix{clone}()} is a reasonable task.}

\sbeffect{Class Extension must be carefully considered}{When a subclass overrides a
significant percentage of (especially all) methods in a concrete non-Adapter Class, it
can easily introduce bugs if the base Class has a method added.  (As a note of
clarification, Adapter Classes provide only empty implementations).  This is an
appropriate location for an Interface.}

\lsubsection{Implement Early Error Checking}

Error checking should take place as early as is programmatically possible.

\sbeffect{Check Errors in Tags}{All error checking should take
place in the Tags/Tokens,\index{token} and those should
load known-good (and \ix{type safe}) information into the PObjects.}

\sbeffect{PObjects must catch AND report illegal loads}{The set
methods of the PObjects should throw
\texttt{IllegalArgumentException} if
invalid information is provided (such as a \texttt{null})}

\sbeffect{PersistenceLayerException is for the \ix{Persistence}
Layer}{\newtextPointZeroFour{The PersistenceLayerException
should be reserved for the \ix{Persistence} Layer, and not used
within the core code of PCGen.}
}

\lsubsection{Minimize Error Suppression and Gating}

Because of Assumption \ref{Implement Early Error Checking},
frequent testing of \texttt{null} should be prevented
(unless \texttt{null} is
explicitly legal due to lazy instantiation or as an identifier).

\noindent\textbf{Corollary} If you think code is unreachable, don't return a \texttt{null}
or \texttt{``''} or just say \texttt{//Can't get here}. 
Put a \texttt{throw \ix{UnreachableError}(``If we got here, the factory
failed to produce a singleton per key'')} (for example), because if the code really
DID get there, we have a serious bug that should be addressed.

\sbeffect{Fixing NullPointerExceptions}{If a dereference throws a
\texttt{NullPointerException}, then the bug fix is
NOT to gate the call with a \texttt{null} check to suppress the error.  NPEs usually
indicate bugs.  The solution is to go back, determine what loaded a
\texttt{null} into the core, and prevent the NPE by preventing a bad
load (but be careful about 
just gating it there, as well, unless the \texttt{null} really is legal\ldots
if not, throw an \texttt{Exception} or do a \texttt{Logging.errorPrint} and return
\texttt{false} if you're in the Tags).}

\sbeffect{Catching Exceptions}{Never catch \texttt{Exception},
\texttt{RuntimeException} or \texttt{Throwable}.  
This should be enforced with \ix{PMD}, \ix{FindBugs}, or a similar tool.}

\lsubsection{Type safety is good}

\index{type safe}
\index{enumeration}
\newtextPointZeroTwo{
Type safety provides a number of advantages.  First, it can minimize the 
number of problems produced in the code due to misinterpretation of what
a String object actually contains.  Second, equality can be tested 
by \texttt{Object.equals()}, which simply tests \texttt{a == b}.  This is
a very fast operation (one method call and one instance equality).
}

\revisedtextPointZeroFive{
\index{type safe}Type Safe Enumerations should be used for Attack Types, Regions, Spell Schools,
Types of Vision, and soooooo many other things (these will be either Java 1.4
or Java 1.5 style) rather than Strings.  This is also not completely limited to 
true typesafe \ix{enumeration}s, as Tom also believes we should be storing \ix{Formula}
in a type-safe fashion, so that they are less likely to be confused with 
general Strings (although this is a VERY minor detail in the grand scheme).
Note also that these Type Safe Enumerations cannot all be Java 1.5 style Enums,
as some of them will require on-the-fly addition of items during \ix{LST file} import
(e.g. AbilityCategory).}

\newtextPointZeroTwo{
Type Safety\index{type safe} also can provide additional
performance advantages.  Acquiring
a type safe reference from a String (which is what is required 
during \ix{LST file} import) does not provide savings over direct String comparison
(and actually provides a very slight loss in the cost of additional 
method calls).  However, when a \ix{type safe} reference is required from core
code, that \ix{type safe} reference can be preconstruced and referenced 
directly.  This trades a Map lookup (slowish) for a direct reference
(fast). \newtextPointZeroFour{
It may also reduce the String constant pool memory overhead of a program.
} Also, when using a \ix{type safe} constant as a key to a Map, the
internal map structure can be optimized.\footnote{This performance
boost is demonstrated by the \texttt{EnumMap} Class in Java 1.5}
The use of a \ix{type safe}-aware Map can guarantee O(1) performance 
for a \texttt{get()} from the Map.\footnote{A \ix{type safe} map can
be constructed to guarantee no collisions will occur in the
key mapping method that is used}
}

\lsubsection{Avoid side effects}
Side effects are confusing, and should be avoided.  For
example, if a get is performed, it should perform a GET.  It should not
trigger other events, including the setting of additional variables. 
Classes should also respect their Interface or super class and implement
that interface without side effects.  A \texttt{FileFilter},
for example, should
simply return true or false, it should NOT process the data that is
received as parameters.  No offense meant to the original programmer,
but this method:

\noindent\texttt{private void loadPCCFilesInDirectory(String aDirectory) \\
\{ \\
\hspace*{0.25in}new File(aDirectory).list(pccFileFilter); \\
\} \\
}

\ldots{}looks like it could (and should) be deleted, but the \texttt{pccFileFilter} is
performing side effects.

\lsubsubsection{Avoid validation based change}

\newtextPointFour{
One architectural issue that can arise is the use of a validation step to perform
cleanup (or changes) in a data structure.  While this is often a noble effort to 
ensure that undesired states are not created, some effort should be made to avoid
relying on the cleanup performed during validation.  (This is a specialized form
of avoiding side effects)
}

\newtextPointFour{
Stated another way, the code should not change make the first part of a change to a 
PC (such as removing an Ability) and subsequently call a ``validate'' method in order
to perform any changes to the player character from that removal.  The removal should 
either be atomic (preferred), or the removal operation should be predictable,
understandable and bounded (if the removal operation requires a full data model traverse 
to perform the effective equivalent of a global validation, it is still violating 
this code recommendation).
}

\lsubsection{Internationalization}

\revisedtextPointZeroTwo{
One of the existing challenges in the PCGen code base is that it was not developed
for use in multiple languages, generally it is not `Internationalized'.  Part of
the conversion to a CDOM environment will provide the opportunity to ensure that
the core components of PCGen are properly internationalized.  It will be a 
prerequisite for code that is checked into both the library (\texttt{pcgen.base}) 
and the new core (\texttt{pcgen.cdom}) to be internationalized.  This is
recognized that it may directly effect enumerations, and how they are 
constructed.
}

\sbtom{Enumerations and Internationalization}{\newtextPointThree{It is noted
that it is possible to use enumerations in an internationalized environment; 
this may require the addition of a \texttt{.getDisplayName()} method to the 
enumeration (to fetch how it should be displayed), but there is not a 
fundamental issue with Java 1.5 enumerations and Internationalization.
}}

\lnsection{Additional Coding Recommendations}

The following are additional, non-architectural coding recommendations for
PCGen 6.0:

\lsubsection{Write Simple Code!}

\revisedtextPointZeroThree{
Any coder can make a complicated piece of code that appears to be really clever.
It takes a higher degree of expertise to make that code simple.  Highly complex
methods, generally defined by cyclomatic complexity - those methods with
many (10+) decision points (if, else, for, while, etc) - should probably be
split into multiple methods \textem{regardless of whether any of those functions
are called from more then one point.}  In general, every method should
be easy to understand sans comments (not to say that there shouldn't be
comments), otherwise it's probably not simple enough.
}

\lsubsection{Ease Code Reuse}
Reuse of code should be clear and straightforward.  This can be facilitated
by straightforward package use.

\revisedtextPointZeroFive{
Passing in control mechanisms as parameters (i.e. booleans that control an if statement)
are a sign that you have a candidate for refactoring into a couple of methods. It won't
always be possible to split the method up, but it can be a useful exercise to try anyway.
Two (or three) simpler methods are generally better than one complex one.
}

\newtextPointZeroFour{
In order to show some code that demonstrates a boolean control structure that should
be refactored to be two methods, the following example is condensed from
a real method in \texttt{LstSystemLoader}:\\
\texttt{
\noindent private PObject methodOne(boolean fi, \ldots) \{ \\
\indent if (fi) \{ \\
\indent ~~\ldots \\
\indent ~~return \ldots \\
\indent \} \\
\noindent \ldots \\
\noindent \} \\
}
}

\revisedtextPointZeroThree{
Some of this is about testability and \index{unit test} unit tests
as much as it is about reusability - e.g. if
one can write tests using Integers as the graph nodes, it's a heck of a lot
easier to write tests, to perform tests and to debug failing tests than
if the programmer is forced to use PObjects in simple tests.
}

\sbeffect{Extract General Code}{General code �- that which does not have a strong association
with PCGen -� should be placed into a \ix{library} of sorts, away from the main
PCGen code.  This should generally be organized in the same sub-packages
as the main Java packages (\texttt{*.lang}, \texttt{*.util},
\texttt{*.io}, etc.) based on what Class
in the main package they are overriding or supporting.  
\revisedtextPointZeroFive{
For example, DoubleKeyMap belongs in a \texttt{base.util} package,
much of CoreUtilities should be split into StringUtil, DoubleUtil, etc. Classes
and placed into a \texttt{base.lang} package,
the graph code should be present in a \texttt{base.graph} package,
EntityEncoder should be in a \texttt{base.language} package.
}}

\lsubsection{Beware long if/then/else, especially with instanceof}
Significant testing of \texttt{instanceof} or significant casting
is an indication of code in need of review.  Double Dispatch should be
used in instances where one might otherwise use \texttt{instanceof}
testing.\footnote{If you don't know what Double Dispatch is, Wikipedia is
a good starting point: \url{http://en.wikipedia.org/wiki/Double_dispatch}
\ldots Note that we are also implementing Double Dispatch in getCheckList of
the ProhibitedSpellType enum if you would like to look at an example.  The Double Dispatch in that case 
is one dispatch to the ProhibitedSpellType, and the second dispatch to the appropriate
method of Spell.  Double Dispatch in this case is used to avoid a if-then test of the
ProhibitedSpellType to determine the proper Spell method to fetch the appropriate List.}

\noindent\textbf{Corollary} Large if then else constructs that make more than 4 or 5
string comparison during \ix{LST file} import probably also need review. 
\revisedtextPointZeroFive{Almost every case of a really long (6+) if-else if
chain should be culled from the codebase sooner rather than later.  
}

\noindent\textbf{Corollary} Large switch statements, too.

\lsubsection{Enforce Consistent Object and Interface behavior}
\texttt{.equals()}, \texttt{.hashCode()} and \texttt{.compareTo()} should act as
appropriate for the entire object, not a subset of its contents.  For example,
\texttt{.equals()} should not solely test on \texttt{.getName()} being equal,
but rather the entire contents of the Object being equal.

\noindent\textbf{Corollary} Realistically, none of those methods should be required in
CDOM objects, since \texttt{.\ix{clone}()} will only be used when an object's name is
changed (see Assumption \ref{Minimize PObject Cloning} below)

\sbeffect{.equals presence}{\revisedtextPointFour{While one might desire the
implementation of \texttt{.equals()}, \texttt{.hashCode()}
and \texttt{.compareTo()} to be avoided in the CDOM code base,
there is still good reason to implement it.  The comparison of two
graphs (used in testing) is most easily done by a \texttt{.equals()}
on the graph.  Thus, implementation of \texttt{.equals()} and 
\texttt{.hashCode()} will still be done in the CDOM code base, but 
it will a true, deep equality, not a key-based equality as it exists
in the 5.10 code base.}}

\lsubsection{Semantic use should be clear}
We are not currently clear on
reference vs. value semantic Classes and that should be cleaned up as much
as possible.  \revisedtextPointZeroFive{Each
Class (preferably in a fashion consistent with other 
Classes in the package) should be choosing whether to be reference or
value semantic and sticking to that definition.}

\newtextPointZeroTwo{
For those not familiar with the terms, a Reference-Semantic Class is one that
takes in references to Objects (such as a List) and maintains an actual reference
to the List.  A good example of this is a \texttt{HashMap}.  The following is valid code; 
the asserts will pass as being correct. \\
\texttt{
\noindent List list = new ArrayList(); \\
\noindent list.add(Integer.valueOf(9)); \\
\noindent map.put(key, list); \\
\noindent list.add(Integer.valueOf(10)); \\
\noindent assert map.get(key).size() == 2; \\
\noindent List list2 = map.get(key); \\
\noindent list2.add(Integer.valueOf(11)); \\
\noindent assert list.size() == 3; \\
\noindent assert list == list2; \\
}
}

\newtextPointZeroTwo{
A Value-Semantic Class is one that
takes in references to Objects (such as a List) and does NOT
maintain an actual reference to the Object (though it may maintain references 
to the objects contained within a Collection). 
A good example of this is \texttt{HashMapToList}.  The following is valid code; 
the asserts will pass as being correct.  \\
\texttt{
\noindent HashMapToList hml = new HashMapToList(); \\
\noindent List list = new ArrayList(); \\
\noindent list.add(Integer.valueOf(9)); \\
\noindent hml.addAllToListFor(key, list); \\
\noindent list.add(Integer.valueOf(10)); \\
\noindent assert map.getListFor(key).size() == 1; \\
\noindent List list2 = map.getListFor(key); \\
\noindent list2.add(Integer.valueOf(11)); \\
\noindent assert list.size() == 2; // 9 and 10 [not 11] \\
\noindent assert list != list2; \\
}
}

\newtextPointZeroTwo{
Note how the list stored 
within the \texttt{HashMapToList} is protected from any change to Lists that
were passed in or returned.
}

\revisedtextPointZeroThree{
This is not an issue of what choices we have to make generally,
as much as making each Class self-consistent and documented.
}

\revisedtextPointZeroThree{
\texttt{HashMapToList}, for example, should not spend 90\% of its code protecting
it's internal structure and then have one item expose it.  For example,
\ix{PrereqObject}'s setPreReq and
setPrereqList both take what is essentially a value-semantic
object and expose the internal structure (Hey world, I use a List!)
(\texttt{setPreReq}) or make it reference semantic (Hey world, write my
private variable!) (\texttt{setPrereqList}).  A JavaDoc warning is not sufficient\ldots
Those methods should be refactored away for semantic consistency.
}

\revisedtextPointZeroThree{
Note value-semantic is a requirement if one expects an object or
interface to be immutable.  For consistency, one should not expose ANY
internal structure.
}

\lsubsection{Safe Construction}

\newtextPointZeroTwo{
\revisedtextPointZeroFive{
Calling a non-final method from a \ix{constructor} allows modification
of an object by a sub-class in a way that can significantly alter
the operation of the Class to the point of altering the original
intent of the Class. Thus, calling
non-final methods from a constructor should be prohibited.  
}}

\lsubsection{Avoid exposing this from constructor}

\newtextPointZeroTwo{
Another design style that should be avoided is exposing a reference to the object
being constructed (`\texttt{this}') from within the \ix{constructor}
(e.g. this can accidentally be done by
registering an object as a listener while it is being constructed). This should
be prohibited.
}

\lnsection{Key Concepts}

\lsubsection{PObjects}

\newtextPointThree{
Classes, Races, Languages, and other characteristics and attributes of a
PlayerCharacter are stored within PCGen as objects.  Typically, these objects are generally
referred to as PObjects, as they extend the generic PObject Class.
}

\newtextPointThree{
Generally, this is a benefit, as common code can be contained within the PObject and
is thus widely reused within the various data objects.  PObject as a class is not ideal, and
some changes will be made, but the general concept of a universal parent for the data
objects will remain in a CDOM environment.
}

\lsubsection{Tags and Tokens}

\newtextPointThree{
Data that defines a Class, Race, Language, and other characteristics and attributes of a
PlayerCharacter are stored in LST files. These LST files are parsed to import the information
from those LST files into objects within the PCGen program.  Generally, the contents of PObjects
constructed during LST file import are defined by Tags within the LST file, such as the Language
Common in the Language LST file from SRD 3.0:\\
\texttt{Common	TYPE:Spoken}
}

\newtextPointThree{
The Java Classes that define the legal tags are loaded as plugins to PCGen.  These persistence
Tokens are non-abstract Classes that implement the LstToken interface. A special map of persistence
Tokens is generated when PCGen launches.  This `special map' is effectively a Map with two keys
(or a Map to a Map for those who want to know the technical structure in the code). The first key
is the Interface implemented by the Token. This identifies which file types the Token is capable
of processing. The second key is the String version of the Token (as it will be found in an LST
file).  This String version is the exact String that will appear in an LST file before the first 
colon `:' in an LST file.  With those two keys, the actual object which can parse the String from
the LST file into the internal representation in PCGen can be resolved.
}

\newtextPointFour{
Note that Prerequisites are actually an exception to the tag lookup system shown above.  The
PRExxx tokens are handled in a separate lookup, which operates in a similar fashion to the token
lookup described above.
}

\newtextPointThree{
By keeping each Token in an individual class, this keeps the Token Classes very simple,
which makes them easy to test, modify, and understand (as they are effectively atomic to the
processing of a specific token).  One goal of CDOM is to ensure that all of the parsing of LST
files is done within the Tokens and not in the core of PCGen.  This makes adding new tags to
the LST files to be reasonably painless (though changes to the core or export system will also
be required to add required functionality).
}

\newtextPointFour{
Note one requirement in token construction is that they must be order independent.  This impacts
a number of interdependent tokens (such as FOLLOWERS and COMPANIONLIST), which means those items
must defer some processing (or dereferencing) until runtime.  Order independence should be considered
an architectural requirement for token construction.
}

\newtextPointThree{
Note that this document is NOT careful about using the terms Tag and Token properly.  They are 
often interchanged, and while it could be argued that it makes the document unclear, the 
distinction is so subtle that maintaining two terms is probably just as bad of a transgression.
}

\newtextPointThree{
Generally, it should be recognized that there are five types of tokens.}
\newtextPointFour{
These token types are listed in the order of increasing complexity, and therefore, they are 
listed in the preferred order for development of new features (preferred is first).  In other
words, when implementing a new feature, if it can be done with an Association Token or as a
List Modification Token, the Association Token implementation is preferred.
}

\lsubsubsection{Primary Information Tokens}

\newtextPointThree{
First, informational tokens in an LST file provide direct and simple information about
the object stored in the LST file.  These characteristics may be as simple as integer values
(such as a cost), strings (such as a Display Name), typesafe values (such as Region), or other 
values.
}

\lsubsubsection{Association Tokens}

\newtextPointThree{
Association tokens in an LST file create an association of some form (generally allows, contains, 
or grants) between the object stored in the LST file and another primary object (defined in the
same LST file or another LST file).  There may be additional information (such as Prerequisistes 
or casting cost) that are part of the association between the parent and child object.
Tokens that produce this effect include FEAT, SPELLS, and CLASSES.  (Note the explicit use of classes 
indicates that the association that is formed could be an association that defines the current
object as either the parent (in the case of FEAT) or as the child (in the case of CLASSES).
}

\lsubsubsection{List Modification Tokens}

\newtextPointFive{
List modifications tokens are those items which can modify a List of
CDOMObjects to add to the contents of a list.  The most obvious of these
are the SPELLLEVEL:CLASS and SPELLLEVEL:DOMAIN Tokens, which (surprise)
modify SpellLists.
}

\deletedTextPointFive{
\revisedtextPointThree{
There are some tokens that grant an association between two other objects.  These third 
party tokens are more challenging to implement and require more complex handling in any
data structure.  An example of this token is MONCSKILL,
\revisedtextPointFive{in which a Race (effectively the 3rd party)
adds a class skill (the 2nd party) to a Class (the 1st party)}.
}
}

\lsubsubsection{Replacement Tokens}

\newtextPointFour{
Some tokens change a Type, Hit Die, Weapon Proficiency, or other characteristic of an object.
These tokens must be handled differently than a Primary Information Token, and a Modifer is
placed into the PlayerCharacter when these tokens are encountered.  One example of a Replacement 
Token \revisedtextPointFive{is the HITDIE token for Templates (which changes the Hit Die
of class levels for the PC)}
}

\lsubsubsection{Factory Tokens}

\newtextPointFour{
At this point, no Factory tokens may be required, but the concept is highlighted in order to
help people avoid the use of the Factory construct.  Factory tokens are tokens where the effect 
cannot be predicted when the token is created; thus, the processing of the token needs to be 
deferred until the object containing the factory token is added to a Player Character.  The 
token most likely to require a Factory is the ADDLEVEL token from the Race LST file.  Since the
levels to be added cannot be predicted from the token, it is likely that the exact processing
must be deferred until the Race is added to a Player Character.
}

\lsubsection{Prerequisites}

\newtextPointThree{
Another important concept in PCGen is the Prerequisite.  These are limitations placed onto PObjects
in order to constrain the ability of a PlayerCharacter to use the PObject (e.g. to limit selection of a Feat
to a minimum Character Level).  There are a few key concepts that should be recognized when dealing
with Prerequisites:
\begin{itemize}
\item Prerequisite resolution must take place when a query for information is desired (Testing a
Prerequisite is not a static calculation, due to temporary changes that may take place with a PlayerCharacter)
\item Objects may not have a Prerequisite individually, but only in a certain context (A Feat may
have a certain Prerequisite only when it is granted by a specific PCClass)
\end{itemize}
}

\newtextPointThree{
Prerequisties are requirements for something to be active on a PlayerCharacter.  For example, the Dodge
Feat (from SRD) would have \texttt{PRESTAT:1,DEX=13} This Prerequiste means two things: 1) a PC must have a
Dex of 13 at the level-up event to select Dodge as a Feat and 2) The PC must have an active Dex of 13
(think temporary changes from items, curses and the like) to be able to use the benefits of the Feat.  Thus,
a critical concept: Prerequisites are not all static, and cannot all be resolved before Objects are
placed in the PlayerCharacter.
}

\newtextPointFour{
Prerequisite resolution must take place when an object is added to a PlayerCharacter.  The result of this
order of operations (processing when added to a Player Character, and not when a query for information 
contained within the Player Character is performed) means that the Player Character will likely need to 
maintain both a complete graph (indicating everything the PC may be eligible for) and an active graph
(indicating the items the PC actually met the Prerequisites for, and thus indicating items the PC was
granted).
}

\sbtom{Prerequisite Order of Operations}{This needs to be heavily tested.  This assumption of Prerequisite
testing on adding to the PC is dangerous, given the current state of the code (it may be true, but the
`business rules' of Prerequisiste testing are really in the code.}

\newtextPointThree{
Also, Objects may not have a Prerequisite individually, but only in a certain context.  Rogues,
for example, are granted certain Weapon Proficiencies. Some of these are dependent on the Size
of the Rogue (from SRD 3.0's hidden feats):\\
\texttt{AUTO:WEAPONPROF|Club|Crossbow (Heavy)|\ldots|[PRESIZEGTEQ:M]}\\
These are only given to a Rogue that is at least size Medium.  This should not restrict anyone
who receives those Weapon Proficiencies, since this is not a characteristic of the Weapon
Proficiency but a characteristic of the Class (literally granted by a hidden feat if you go
looking through the LST files).  Therefore (Prerequisite key concept \#2): Prerequisites can also
be a part of associations between objects.
}

\lsubsection{Relationships}

\lsubsubsection{Grants Relationship}

\newtextPointFour{
An object (meaning a PObject, such as a Class, Ability, or Race) can `grant' another object.  This 
relationship indicates that the benefits of the granted object are received by the Player Character 
when the granting object is added to the Player Character.  For example, the Bard class, at first level,
grants the Ability ``Bardic Countersong''.
}

\lsubsubsection{Allows Relationship}

\newtextPointFour{
An object (meaning a PObject, such as a Class, Ability, or Race) can `allow' another object to be
granted, but not directly grant that object.  This relationship indicates that the benefits of the
allowed object are NOT received by the Player Character when the allowing object is added to the
Player Character; however, access to select the allowed object is granted by this relationship.
For example, the Dwarf race allows access to the Language ``Giant'' as a starting Language.
}

\lsubsubsection{Contains Relationship}

\newtextPointFour{
An object (meaning a PObject, such as a Class, Ability, or Race) can `contain' another object.  This
indicates that the benefits of the allowed object are NOT received by the Player Character when the
allowing object is added to the Player Character.  This relationship solely indicates ownership; it 
does not convey any capabilities.  The best example of this relationship is that of a pocket (which
may contain other objects, but does not grant any capabilities of those objects).
}

\lnsection{Proposed PCGen Object Architecture (``CDOM'') Characteristics}

First, it is appropriate to make some additional CDOM specific assumptions
in order to guide the discussion of the architecture of CDOM.  Therefore:

\lsubsection{Minimize PObject Cloning}
Observation: Currently the PCGen code base is forced to frequently \ix{clone}
PObjects in order to use those objects within a PlayerCharacter.

Cloning a PObject should only be required when a .COPY tag
is struck in the \ix{LST file}s.  A PObject should never require cloning in order
to use it in a Player Character.

\lsubsection{PObject should be light-weight}
Observation: Even with Assumption \ref{Minimize PObject Cloning} above,
there will still be a significant number
of PObjects present in memory at any one time.

An information-light PObject should be very lightweight in
memory (a Language should fit in \textless 256 bytes; ideally
\textless 128) 

\lsubsection{Minimize Level Awareness}
Observation: Too many Classes in the PCGen code base today are level-aware. 

Level should be implicit in the character structure and handled as an
exception, not passed to objects that should otherwise be level ignorant.  Said another
way, only PlayerCharacter and a PCClassLevel factory (may be PCClass) should be level
aware.  PObject, SpellSupport, Tags, and ALL other Classes should be level ignorant.

\revisedtextPointZeroFive{
As a noted use case, there are objects granted by Templates which are granted at
a specific level.  If a Skill is granted at level 4, the Skill Class is level-ignorant.
The Skill Class, however, has Prerequisites, and there will (of course) be a 
\ix{Prerequisite} that is level-aware.  This design statement (minimizing level awareness)
is primarily aimed at removing level awareness from:
\begin{itemize}
\item LevelAbility* (actually these are entirely eliminated by this proposal)
\item Tokens that have Level passed in (PCClass Tokens)\index{token}
\item PObject
\item SpellSupport
\end{itemize}
}

\lsubsection{Completely remove token knowledge from core}

Observation: Today, \ix{token} knowledge is contained with the core Classes in two ways: 
First, while progress has been made to eliminate it, some token input processing takes
place in the core.  Second, token \ix{output} processing is always handled in the core of
PCGen 5.10.*, as the tokens are \ix{output} in the various \texttt{getPCCText()} methods.

Output should be implicit in the tokens, and the interfaces of the Tokens expanded to 
require both a \texttt{parse()} and an \texttt{unparse()} method.  This completely 
extracts token processing from the core, and ensures that the parsing and unparsing are
present in the same Class.

This facilitates easy \ix{unit test}ing (a parse/unparse cycle should 
not change the information).  This is also aligned with Assumption
\ref{Contracts are prohibited}, which prohibits update knowledge (\ix{contract}s).
The complexity of 
\texttt{getPCCText()} can easily result in a PObject that is no longer properly unparsed
when new variables are added to PCGen.  Note that new variables may be added simply by
adding a new \ix{token} \index{plugins}plugin, and the core PObject need not be modified.

\newtextPointZeroFour{
To facilitate the \texttt{unparse()} operation, each reference in the master data imported 
from the \ix{LST file}s will contain the Token\index{token} name that drove creation of the reference. This allows
appropriate ``unparsing'' to occur on a large scale (such as \ix{output}ting the entire \ix{LST file}
structure) in an automated fashion without great complexity.
}

\newtextPointFour{
Note that the support of .CLEAR in a parse/unparse cycle requires additional information beyond
the resulting value of a Token.  This drives complexity penalty that is undesirable for the runtime
environment. Therefore, the internal data structure used by the LST editors will be different 
(just slightly more complicated) than the internal data structure used by the runtime environment.
}

\sbtom{On Tokens}{\newtextPointThree{I believe this also means we want to have the tokens somehow specify how 
they should be edited (in an LST editor).  There are some tradeoffs here.  I would prefer to 
keep the Tokens generally GUI-unaware, so that requires either a non-UI response (something that
might be VERY hard for the complex tokens) or a separate Plugin that has the ability to 
specify how the Token is edited.  This allows for advanced features in the Editor plugin, 
it allows the implementor of new non-PCGen tokens to ignore the editor (but the editor
can still include the token through a default editing style) and it generally removes the plugin 
information from the core (since if the tokens are not in the core, then the ability to edit them
should also not be in the core).
}}

\lnsection{Proposed PCGen Object Architecture (``CDOM'') Benefits}

\lsubsection{Code Clarity}

\newtextPointZeroFive{
Since the new code introduced will be cleaner than the existing code, a significant number
of warnings that exist from tools like \ix{PMD}, \ix{FindBugs}, and many others will not exist.
This will help to highlight actual problems that may exist in the PCGen code.
}

\newtextPointZeroFive{
Enforcing the package relationships will make code modification much easier, since any
given change will impact fewer Java Classes.  Code stability (~35\% in PCGen 5.8.* and about
60\% in the current 5.11.* alphas) should increase dramatically\ldots target 90\%.
}

\lsubsection{Improved Test Coverage and Capability}

\newtextPointZeroFive{
Eliminating tangles in Class dependency will improve the ability to write true \ix{unit test}s
(which are tests that work on a single Class).  This means it will be possible to catch 
smaller errors due to incorrect modifications of the PCGen code base.  The existing ``unit 
tests'' may remain as \ix{unit test}s (e.g. the Prerequisite tests).
}

\newtextPointZeroFive{
Note the existing regression tests (e.g. the pcgen.gui.* tests) will break in the new structure
because .pcg file import will not trigger creation in the same way in a CDOM environment.  
Therefore, a new regression test series (and perhaps a facade to assist in the transition) will
be required.
}

\newtextPointZeroFive{
Clarity in our architectural goals and standards should allow us to generate a series of tests:
\begin{description}
\item[architecture tests] Did someone violate any architectural rules, such as: using ``test'' 
in a non-test method or Class name, accessing the PCGen core from the library code, or accessing
a plugin directly from the core.
\item[\ix{unit test}s] Testing one Class at a time.
\item[integration tests] Test multiple Classes together.
\item[regression tests] Develop a suite of defined full-processing runs that create and output
characters.
\item[performance tests] That check how long it takes to perform certain operations (may be as 
simple as tracking the time taken to perform the regression tests).
\end{description}
}

\newtextPointZeroFive{
Better organization will also
allow \ix{GUI}s to be tested and a ``qualification routine'' could be created to ensure that
any GUI that is built is prepared to respond to the events that PCGen sends and receives.
}

\lsubsection{Ease of Changes}

\newtextPointZeroFive{
It should be possible in PCGen 6.0 to add entire functionality through a new Token without 
modifying the core.  Creation of an LST import Token\index{token} and an \ix{export} token 
should be sufficient to add simple function.
}

\newtextPointZeroFive{
This may actually lead to requests for a Token manual, in order to define how to write tokens
for other systems that the base PCGen may not ship with.
}

\lsubsection{Increased Flexibility}

\newtextPointZeroFive{
While this has not been directly discussed elsewhere in this document, the increased 
flexibility (by eliminating a significant number of special cases that currently exist 
in the PCGen code base) will allow PCGen to more easily support non-d20 systems.
}

\lsubsection{Fewer Classes}

\newtextPointZeroFive{
This architecture should result in fewer Classes in PCGen.  If it does
not, it will be a result of significantly smaller Classes than the PCGen 5.11.* series contains.
}

\lsubsection{Performance}

\newtextPointZeroFive{
This architecture should lead to better performance.  If nothing else, it will make solving 
performance issues much easier, as the acyclic nature of the package structure will make it 
far easier to determine the culprit in any performance issue.
}

\lsubsection{Undo/Redo}

\newtextPointZeroFive{
PCGen in a CDOM environment will acquire Undo/Redo capability.
}

\lsubsection{Campaign Source Validation}

\newtextPointThree{
Since all references will be resolved at load time, this will ensure that:
\begin{itemize}
\item All objects in a given namespace possess a unique KEY, regardless
of the source file
\item All object which references other objects in their tags are referencing
an object that was actually constructed and loaded
\end{itemize}
}

\newtextPointFour{
Different levels of warnings and errors on source load will be required.  The challenge 
is that a \texttt{!PRECLASS:1,Fighter} shouldn't produce the same level of input warnings as
\texttt{FEAT:Dodge} if the Fighter Class and the Dodge feat are not loaded, respectively.  The
reason for this is that the PRECLASS may simply be gating on an item not present in the current
load (which isn't a problem, since it's an inverted Prerequisiste; thus the Prerequisite should
always pass).  The FEAT token is granting the Dodge Feat, however, so if the Dodge Feat is not 
loaded, that should produce a more significant error message.
}

\lnsection{Summary of the architecture proposal}

\lsubsection{Exploring Alternatives}

\newtextPointFour{
Evaluating the architectural goals leads to a constraint on the structures that
can be used for the PCGen data model.  The current core uses a (slightly complicated)
list model (each PObject contains a List of the PObjects it grants or allows).  This
is undesirable for a few reasons.
}

\newtextPointFour{
First, a model based on lists requires validation-based change.  Given that this concept
is only marginally understandable without an example, here
is an example of validation-based change.  Assume the following:
\begin{itemize}
\item Ability A1 grants Ability AMultNo
\item Ability A2 grants Ability AMultNo
\item Ability AMultNo may only be granted once
\item Ability AMultNo grants 2 skill points to the PlayerCharacter
\item PlayerCharacter MyPC has both Ability A1 and Ability A2 (chosen in that order)
\end{itemize}
}

\newtextPointFour{
One possibility would be to have A2 NOT contain AMultNo (since it already exists in A1)
when placed into MyPC.  This would require that the list model is constructed so that
the PObjects themselves are not aware of the relationships (otherwise, the PObjects would
require cloning as they are placed into a PlayerCharacter, and we are back to the PCGen
5.x core).  This is possible, but there are still problems with this model.  If
A1 is subsequently removed (due to a PC change or due to a REMOVE: Token), then the
validation of the PlayerCharacter must realize that the Ability A2 should then grant
AMultNo.  Thus, the validation that is performed on MyPC must alter the internal data
model if A2 does not contain AMultNo in the given example.
}

\newtextPointFour{
Another possibility would be to suppress the expression of AMultNo from Ability A2.  
This does little to simplify the situation if A1 is subsequently removed.  The
validation of the PlayerCharacter must still realize that the Ability A2 should grant
AMultNo (and thus the validation must reactivate the Ability).  While there may be a 
solution to that problem that does not require validation-based change, it is not an 
ideal solution.
}

\newtextPointFour{
Another point to make about list models is the required time to search the model.  For 
an Ability (Feat) that has associations (such as Simple Weapon Proficiency), a full
depth search of the list would be required to extract all of the places where Simple
Weapon Proficiency was referenced to ensure that all assocaitions are evaluated.  
Determining skill levels is equally difficult, requiring a full list traverse to find
all of the references to the given skill.  The underlying issue with this data model 
is that an object can be stored in more than one location in the PlayerCharacter.  Thus,
the entire PlayerCharacter must be searched each time an association must be evaluated.  
That search must also be complex enough to resolve situations where an object can only
be granted one time (MULT:No), to ensure integrity.  With all of these challenges,
the list model is postulated (without proof) as being a problem for performance and clarity.
}

\lsubsection{CDOM is a Graph}

\newtextPointFour{
Given the discussion of alternatives (above), it should be recognized that the}
\deletedTextPointFour{The} association between PObjects is not a pure tree.  
There can be multiple associations to a given PObject.  For example, a Template 
and a piece of Equipment could grant a particular Skill.

It is therefore proposed that the internal memory structure of CDOM be a mathematical graph. 
For a detailed explanation of graph theory, Wikipedia is a good start.\footnote{see:
\url{http://en.wikipedia.org/wiki/Graph_theory}}

\mapfig{graphsample.png}{\label{Fig: Graph Sample}Sample Player Character Graph}

At a high level, graphs generally contain nodes (sometimes referred to as vertices) and edges.
The edges provide connections between the nodes.  In CDOM, the PObjects will be nodes. 
Each PObject will contain the static information that is associated with that particular
PObject.  Dynamic information may be stored within another PObject (node) or within an edge,
if that information is a result of the association of two PObjects (there are some ``use cases''
in Section \ref{PlayerCharacter cases and architecture refinement of the PCGen CDOM}
to flesh out some implementation details of CDOM).  
\newtextPointZeroFive{You can see a sample of a 
Player Character PCGenGraph in Figure \ref{Fig: Graph Sample}.  This sample has been pruned to
show the general characteristics of the PCGenGraph while making the view manageable.}

Each Node will be a derivative of a \ix{PrereqObject} (this should not be limited
to PObjects).  This is done because wasting the entire PObject overhead
on a simple object (such as SpellResistance) is bad, especially given the quantity of content
to be attached to the PObjects.

Each Node will be completely ignorant about what it is connected to.  
\newtextPointZeroFive{This means that when a PCClass 
grants a Feat (such as the Cleric PCClass from RSRD granting Aura of Good
[shown in Figure \ref{Fig: Graph Sample}]), the only association between PCClass and Aura of Good is
the edge within the Graph.  PCClass does not maintain a reference to ANY other PObject.}\footnote{Actually,
PCClass is probably a bad example here, since it will likely maintain references to PCClassLevel objects.}

\newtextPointZeroFive{
In the case of the PCClassLevel to PCClass relationship (and hopefully only in that limited case),
there will be required associations between Nodes that will require a parent Node to
be removed from the Graph if a Child is removed.\footnote{This is really non-sensical in what PCGen
would really allow to take place, but the point is to consider `theoretical' actions on the Graph
and how PCGen should resolve them to avoid an inconsistent state}. This association will involve a
special edge (depicted as an edge with three lines in Figure \ref{Fig: Graph Sample}) that 
will trigger the removal of the parent if the child is removed (but not the opposite, since removal
of PCClassLevel (Cleric, 2) should not trigger the removal of PCClass (Cleric)).
}

Each edge will contain a reference to the source Node and sink Node to which it is connected.  
\newtextPointZeroFive{The edge may also contain additional information about the association between
the two PObjects.  This assocation could be as simple as a prerequisite associated with the acquisition
of the child PObject (also displayed by Aura of Good, shown in Figure \ref{Fig: Graph Sample}).  It may
also be a SubSelection made in a Feat, Hit Points for a Hit Die object, or cost for an Equipment Modifier.
Certain Edges will provide the ability to have a `weight' so that an object which provides 
multiple levels of a Skill can avoid multiple edges connecting the same PObjects (multiple identical edges
would clutter the graph).
}

A master PCGenGraph object stores the list of Nodes and Edges in the Graph, and can be queried to 
\texttt{getAdjacentEdgesToNode(Node n)}.  These associations will likely be cached in our Graph
implementation.

\sbeffect{Recursive Subroutine calls not possible}{Contrary
to prior discussions on the PCGen-dev mailing list, 
one side effect of the Nodes being ignorant of what they are 
connected to is that querying the PCGenGraph cannot be done as a recursive subroutine call.
Such a query for information will have to be done through a graph traversal.}

When \ix{LST file}s are imported, a master PCGenGraph will be created.  This graph will not have
a defined root node, but will be a series of nodes with the connections defined by the \ix{LST file}.
Each connection between the PObjects in the PCGenGraph will be a directional edge (The
``Winged'' Template gives the character ``Movement: Fly (30')'').  Mathematically, it is therefore
a Directed Graph.

\newtextPointThree{
Previous versions of this documented specified that the PCGenGraph would remain acyclic.  This is
not possible with the proposal in the current form.  There are at least one good case that cause cycles:
Synergy bonuses.
}

\deletedTextPointThree{
In order to prevent problems (and actually catch problems in \ix{LST file}s), it is recommended
that the PCGenGraph also be an acyclic graph.  It is possible that cycles are necessary, but
those could result in infinite loops in the code.  Without a very specific use case that
justifies cycles, they can and should be highlighted as an error during \ix{LST file} import. 
Therefore, the PCGenGraph is a Directed Acyclic Graph.
}

\revisedtextPointZeroFive{
\index{bonus}Bonuses to Skills will be one area where we must consider the value of putting the edges
in the graph vs. simply storing as a number and requiring a Skill query and then a bonus
query to actually get the active Skill level.  The fact that Epic modifications to Jump
and Tumble are formulae imply to me that we must leave them that way,
as they are effectively impossible to decompose into discrete edges.  Thus, these items
do not pose a cycle risk in the Graph.
}

A PlayerCharacter will also contain a PCGenGraph (though an independent instance from the
master PCGenGraph, it will be the same Class).  This graph is composed of the objects that have
been given to or selected by the Player Character.  When a PObject is added to a PlayerCharacter,
the added PObject is referenced in the master PCGenGraph.  The added PObject and its entire 
sub-graph are then grafted into the PlayerCharacter PCGenGraph.

Each object (such as a Template) will then reference the additional capabilities that are
provided to the PlayerCharacter.  The PCGenGraph can be queried for the capabilities the PC.

In order to facilitate \ix{unit test}ing and minimize memory usage, every PObject will be ignorant of
all other PObjects.  ALL associations will be made in the PCGenGraph.  The PCGenGraph is
capable of providing:
\begin{itemize}
\item A full list of all PObjects contained within the PlayerCharacter
\item Connections (edges) associated with any given PObject (contained within the PCGenGraph)
\item How a particular PObject came to be associated with the PlayerCharacter. (e.g. Chose Combat
Reflexes as a feat at third level and Weapon Focus (Dagger) as a first
level fighter bonus feat.)
\end{itemize}

\newtextPointFive{
It should be noted that there are specific situations that force the PCGenGraph to be an identity-based
Graph (at least for the edges).  Consider one example: If ADD:FEAT|2|ANY were to be used, then the 
PlayerCharacter could select the same Feat twice (if the Feat was MULT:YES).  Since the management of
edge weights in this situation would be significant (there are other cases where the edge weights can 
be based on a formula, not an Integer), this is best accomplished with two equal, but identity-independent 
edges. (meaning a.equals(b), but a != b)
}

\newtextPointFive{
If edges are not based on identity, then one must find data
that will make the two semi-identical edges unique (but
deterministically unique, so two graphs can still be compared with
.equals).  I believe this would include storing the Source line number,
source column number, and edge request count for the given token in
order to make it deterministic and unique.
}

\newtextPointFive{
The reason so much information is requried is that the source line number required to distinguish:\\
\texttt{CustomDomain1 <tab> FEAT:Bar\\
CustomDomain1.MOD <tab> FEAT:Foo\\
CustomDomain1.MOD <tab> FEAT:Foo\\
}
...while the Source column number required to distinguish:\\
\texttt{CustomDomain1 <tab> FEAT:Bar <tab> FEAT:Foo <tab> FEAT:Foo\\}
... and the edge request count required to distinguish:
\texttt{CustomDomain1 <tab> FEAT:Bar <tab> FEAT:Foo, Foo\\}
Thus it is far easier to simply avoid the issue by making equal edges allowed as long as those edges
have a unique identity (a != b).
}

\lsubsection{Theoretical information is contained in Lists}

\newtextPointFive{
Early revisions of this document suggested that there were multiple edge types that would be
present in the Graph.  That is not currently part of the proposal.  There are a number of reasons 
why ``allowed'' information is not to be stored in the Graph:\\
\begin{itemize}
\item Managing context (for items like Class Skills) would require extremely complicated
edge behavior if it was stored in the graph.
\item Graph traversals are far more complicated when the edges have more than one meaning
\end{itemize}
}

\newtextPointFive{
When an object initializes a list (such as the CLASSES: token in Spell
LST files), they will write to a master list.  When an object modifies
a list, a modification of the list will be stored in the owning object.  This
modification may include associations (such as the spell level,
prerequisites, etc.).  These list definitions and list modifications are
much like variables, in how the tokens DEFINE:Var|value (the definition) and
BONUS:VAR|Var|bonus\_{}val (the modification) operate in PCGen 5.x.
}

\newtextPointFive{
Generally, lists works the same way the variables would.  If a character
doesn't have a DEFINE that defines a variable, then a BONUS to that
variable will have no effect.  Lists will work the same way.  This is a
bit more complicated than Variables/Bonuses (due to the processing time
and memory waste I point out in the Fighter/Spell List example above),
but the principle is the same.
}

\newtextPointFive{
When a character is *actually granted a list* (meaning a List object
appears in the active graph of the PC), a new (empty) List of the
specified type is created for the PC.
The new list is initialized with any data in the master list.
After the ``active object resolution'' process is complete, the list will
be modified to include any objects added to the list by any active
objects in the PC.  
}

\newtextPointFive{
In order to avoid a traverse of the entire graph to grab the contents of
a list (that would be slow and annoying), there will be a method of
consolidating the modifications within a PC.  When an
object granting modifications to a list is added to the active
object graph of the PC, there will be *no effect* on the PC.  This is to
avoid pollution of the List contents of the PC with numerous SpellLists
or other items which are modified by objects possessed by the PC.  (e.g.
If a character is a Fighter and a Template adds a Spell to Arcane
Spellcasters, we don't want to put the Arcane SpellList into the PC -
wastes processing time and memory)
}

\newtextPointFive{
The resolution process operates something like:
Addition of Spell List ``Wizard'' Mod 1\\
\indent No effect on PC\\
\indent ``Active Object Resolution'' takes place due to new objects\\
Addition of Spell List ``Wizard'' Definition (PC is now a Wizard)\\
\indent ``Active Object Resolution'' takes place due to new objects\\
\indent Creates PC Spell List ``Wizard'' from base list and applies Mod 1\\
Addition of Spell List ``Wizard'' Mod 2\\
\indent ``Active Object Resolution'' takes place due to new objects\\
\indent (Re-)Creates PC Spell List ``Wizard'' from base list\\
\indent Applies Mod 1 and Mod 2\\
}

\newtextPointFive{
It is realized that some lists will not be explicitly initialized by any
dataset; nor will they ever be ``added'' to the PC.  These lists will have
keys that start with *, and they will be internal lists to PCGen.  No
initialization will take place, and any object which modifies one of
these lists will trigger an automatic addition of the list to the PC
(the list does not need to be ``granted'').  One example of this is the
*Starting LanguageList, which will be added to by the LANGBONUS token.
}

\newtextPointFive{
It will have to be legal for a List to contain another List (e.g. the
``Aberration'' SkillList [implicitly created by the existence of the
``Aberration'' Class] will contain the ``*Monster'' SkillList [which is the
list added to by the MONCSKILL and MONCCSKILL tokens] due to the
presence of TYPE:Monster in the ``Aberration'' Class)
}

\newtextPointFive{
As a ``bonus'' to this, the graph is actually simpler than in any of the
ideas I presented in the older third party token discussion.  Since there is
no longer an ``allows'' edge (all the theoretical information has been
transferred into lists), the traverses will be faster and the traverse
classes will contain less code.  There is also no risk of accidentally
writing the granted contents of theoretical
information into a persistent copy of the PC... and no indirect
references or code workarounds to avoid the problem
}

\newtextPointFive{
While the Class Skill and Domain/Class Spell Lists will leverage the List system, one
should note that a Class Spell List is distinct from a Spell List.  This is 
required for a number of reasons.  First, it is best in order to avoid namespace
sharing between Classes and Domains (it is now possible to have a Domain and a Class with 
the same name).  This also allows list type safety, which simplifies a few Tokens and 
reduces the chance for error.
}

\lsubsection{Keys may not be part of a PObject}

\newtextPointThree{
Having the Key be independent of the PObject referenced by the Key may be a required part 
of the CDOM Architecture.  (This may also require a bit more thinking to understand what 
breaks and will need to be changed)
}

\newtextPointThree{
Consider the case today, where a Key is an integral component of the PObject.  If one wants
to determine which PObject of a given type is referenced by a given Key, a call to a global
list of objects is made (most of these are stored in \texttt{Globals}) and the entire list
of objects is iterated over to extract the object with the appropriate key.  Simply stated,
this does not scale.  A large number of loaded objects will cause these queries to expend 
significant effort in what should be a simple lookup.
}

\newtextPointThree{
Simply converting Globals to using a HashMap (from Key to PObject) seems like a reasonable
solution, but introduces a dangerous contract: Every change of a Key must be registered 
back to the Global index, because the key is stored both in the PObject and used as a Key
in the Global list.
}

\newtextPointThree{
Simple solution: Make the global list the only place that has the key, and the PObject no 
longer knows its own key.  The only way to change the key is to register a key change
with the Global List.  This has a number of side effects.  The main problem will
be the transition, as the current PObjects often use their key to determine equality.
}

\newtextPointThree{
Another challenge will be in items that allow wildcards (e.g. PRETEMPLATE).   This is 
simply an implementation detail, as the system should be able to do some form of wildcard
match against the keys.  (There are multiple solutions to this, it does not need to 
be resolved here)
}

\newtextPointThree{
Bonuses that are self referencing ($<$this$>$) may require that BONUSes are stored as 
native objects in the graph rather than information.  On the other hand, it will be
impossible to destroy all indirect references (indicating the Weapon Proficiency of a piece
of Equipment is a good example of a place where an indirect reference will be held), so
the references will have to be `taught' the resolved object.  Thus, this also has 
multiple solutions and is an implementation detail that does not need to be resolved here
(although it is recognized as an issue to ensure it is properly handled).
}

\lsubsection{Classes in a CDOM PCGen}

\newtextPointZeroFive{
There are a number of new Class Groups present in this architecture:
}

\noindent\newtextPointZeroFive{The Graph Classes \textem{(and Interfaces)} include:
\begin{description}
\item[Graph Core Elements] include \textem{Graph, GraphEdge, GraphChangeListener, DirectionalGraph,}
GraphChangeSupport, DefaultGraphEdge, ListMapGraph, NodeChangeEvent, EdgeChangeEvent, and others.
\item[Graph Edits] extend UndoableEdit and include InsertGraphNode, DeleteGraphNode,
InsertGraphEdge, and DeleteGraphEdge.
\item[Graph Commands] include DeleteEdgeCommand, DeleteNodeCommand, ReplaceNodeCommand,
PruneUnconnectedToNodeCommand, and other commands used to modify a Graph.  All Commands return
an UndoableEdit capable of `undoing' the change made to the Graph.
\item[Graph Visitors] include BreadthFirstTraverseAlgorithm, DepthFirstTraverseAlgorithm,
NodeDistanceCalculation, DijsktrasAlgorithm, and directed derivatives of the traverse algorithms.
These are the Classes used to traverse the Graph, and will thus be widely used to extract
information from a PlayerCharacter.
\end{description}
}

\noindent\newtextPointZeroFive{The other \ix{library} Classes include:
\begin{description}
\item[Static Utility Classes] such as StringUtil, Delta, BigDecimalHelper, and EntityEncoder
\item[Instance Utility Classes] such as DoubleKeyMap, ListSet, MapCollection, and WeightedList
\item[Universal Classes] such as UnreachableError
\end{description}
}

\noindent\newtextPointZeroFive{The CDOM Classes include:
\begin{description}
\item[CDOM Graph Components] such as PCGraphGrantsEdge and PCGenGraph.
\item[CDOM Core Objects] including \ix{PrereqObject} and the generic ``CDOMObject'' parent of the
CDOM Instance Objects.
\item[CDOM Instance Objects] such as PCClass, PCTemplate, Skill, Spell, and many others.  These may be
any node within the PCGenGraph.
\item[CDOM Content] objects that are contained within a PCGenGraph, but are not primary
CDOM Objects.  These elements will NOT contain outgoing edges in the PCGenGraph (they are
only leaf nodes of the PCGenGraph).  This includes
items such as DamageReduction, SpecialAbility, Movement, and HitDie.
\item[CDOM Enumerations] \index{enumeration}
such as AbilityCategory, AlignmentType, IntegerKey, SpellType, and Visibility.
Some of these enumerations (e.g. Visibility) will be fixed and be Java 1.5 enumerations.  Others
must be flexible with \ix{LST file} input, and thus must be expandable Java 1.4 \ix{type safe} \ix{enumeration}s
(e.g. AbilityCategory).  These Enumerations are (generally) used as keys to access data stored within
the PObjects.
\item[CDOM Helpers] are ``complex'' support Classes contained within a
CDOM Object such as SpellSupport, Source, SpellFilter and BonusSupport.
\end{description}
}

\lsubsection{General Use of CDOM Classes within the PCGenGraph}

\newtextPointZeroFive{
The CDOM Classes within PCGen will be loaded into the master PCGenGraph during the \ix{LST file} import.  
Generally, any `primary' object that has an \ix{LST file} (e.g. Skill, PCClass, Spell), will have
an associated PObject loaded into the PCGenGraph.  
}

\newtextPointZeroFive{
Any references by that object to other 
PObjects (such as granting a Skill from a Template) will result in a Containing Edge being added 
to the master PCGenGraph.  If theoretical information is added (such as granting a
particular Spell), then a Granting Edge is added to the master PCGenGraph.
}

\newtextPointZeroFive{
A PObject may also grant other characteristics that are not PObjects.  These content objects 
(e.g. DamageReduction, SpecialAbility) may not 
be a parent in the PCGenGraph to other content or PObjects.  It is possible
for more than one PObject to grant identical content objects.  All content objects implement
\ix{PrereqObject} and may contain \ix{Prerequisite}s.
}

\newtextPointZeroFive{
\ix{Prerequisite}s can be contained within a PObject, content or an edge.  If a Prerequisite is 
contained within a PObject or content, then the contained Prerequisite(s) must be passed
in order for the PlayerCharacter to have access to the PObject or content.\footnote{``Passed''
in this use includes overrides from QUALIFY and other methods that may be developed to
avoid Prerequisite tests}  If a Prerequisite is contained within an edge, then the Prerequisite
within that edge must be passed in order for the PlayerCharacter to have access to the child 
PObject or content (through that edge - it is possible for the PC to have access to the child 
PObject or content through a different edge if that alternate edge has no Prerequisites or if
the alternate edge Prerequisite(s) are passed).
}

\newtextPointZeroFive{
Similar to how PObject operates today, specific characteristics of a PObject will be stored 
directly within the PObject.  This includes many of the characteristics that are stored today 
using the IntegerKey, StringKey, and related *Key \ix{enumeration}s.
}

While the master PCGenGraph is being built, a \ix{LoadContext} object will
be passed in to each tag.  This LoadContext will be capable of:
\begin{itemize}
\item Grafting new objects with proper associations into the PCGenGraph
\item Removing objects and their associations from the PCGenGraph
\item Generating typesafe keys and storing those in a master list
\end{itemize}
Effectively the vast majority of static Class references can be eliminated from PCGen, which
would allow two independent Campaigns to be loaded at any given time.

\lsubsection{CDOM Lists}

\newtextPointFour{
There are numerous objects that are gathered into lists within PCGen.  It is beneficial to
standardize these lists in order to simplify processing and comprehension of the CDOM
data structure.  These lists may be Skill Lists, Spell Lists, Template lists, Ability Lists,
or lists of other PObjects.
}

\newtextPointFour{
These lists may contain objects bestowed as an `allows' relationship (such as Skill Lists that indicate 
whether skills are CLASS or CROSS-CLASS skills) or a `contains' relationship (such as Equipment Lists that
indicate what is carried by the Player Character).
}

\newtextPointFive{
Items can be conditionally added to Lists, in that they are added only if a Player Character 
passes certain prerequisites.
}

\lsubsection{Structure of a Player Character}

\newtextPointFour{
Due to the order of operations for Prerequisites (see Section \ref{Prerequisites}),
the Player Character will contain two object graphs.  The first graph is the complete graph of
the objects the character could be eligible for.  The
second graph is the graph containing only the active objects for the player character 
(items where the PC met the prerequisites and any equipped items).  Checks for presence
in a PlayerCharacter will generally be made against the active graph.  No check needs to be
performed in that case against any Prerequisites (because those issues were resolved as items 
were copied from the complete character graph to the active graph).
}

\lsubsection{Extracting Information from the Graph}

\newtextPointZeroFive{Extracting information from the Graph requires a number of steps. 
First, a query can be performed to determine if the given object (such as the Skill ``Tumble'')
is contained within the PlayerCharacter PCGenGraph.  If it is not contained within the PC graph,
then the PlayerCharacter does not directly have the Skill.  Checking for BONUSes\index{bonus} may still be
required.  
}

\revisedtextPointFour{
If the Skill is contained within the PC's active object graph, then the paths back to the 
`root' of the PlayerCharacter must be tested.  
Consider that there may be multiple paths, because the Skill might be granted by multiple things 
(by a Class, by a Template, by an Ability).  Therefore edge weight must also be considered 
if the tested object has ranks (like Skills) or can be taken multiple times (some Feats).
In the case of a skill, the combined weight of the edges leading back to the root
indicates the rank of the Skill.
}

\newtextPointThree{
This test may sound complicated... but that's okay, because graph theory is actually
heavily researched and well understood.  We just need to model our problem as something
that has already been solved.  So, really, we can address this as a ``can we get there
from here'' problem, with the source being the PC root and the Skill being the
destination.\footnote{Remember the graph is `directed' with the source of an edge being
what granted the PObject attached as the sink of the edge, thus ``PCClassLevel(Cleric,1)''
grants Feat ``Turn or Rebuke Undead'' by having an edge from the PCClassLevel to the Feat).}
}

\newtextPointThree{
In order to test whether two items are connected in a graph, we can use Dijkstra's
Algorithm.\footnote{See: \url{http://en.wikipedia.org/wiki/Dijkstra's_algorithm}}
Define the edge length as 1 if all the \ix{Prerequisite}s pass and the Node length as 1
if all those \ix{Prerequisite}s pass.  Neither can be traversed if any \ix{Prerequisite}
contained in the respective node or edge fails.  Only traverse edges from source to sink.
Thus, we have a rather simple test to determine if a PC has a given skill.  We can run
Dijkstra's algorithm with the appropriate edge and node traversal rules.  If the length
is non-infinite, the PlayerCharacter has the given Skill.  The traversal can concurrently
calculate (and multiply out) the weight of the traversed edges, providing the final Skill
level.
}

\newtextPointThree{
Actually, from a performance perspective, this may be a bad way to do it, since the
PCGenGraph is high fanout (a PCClass will grant a lot of stuff, but an Ability will only be
granted by a few things).  Therefore, one could actually perform Dijkstra's algorithm
backwards, and use the Skill as the source and only traverse edges from Sink to Source.
The situation is not \textem{quite} that simple, however, so it is likely that PCGen will
use a top-down calculation or a bottom-up calculation depending on the type of information
being extracted from the PC's PCGenGraph.
}

\newtextPointZeroFive{At this time, even non-qualified objects will be placed into the 
PlayerCharacter's PCGenGraph.  This is done to avoid pre-mature optimization.  It is possible
that the PC graph containing numerous objects that the PC is not qualified for will be a 
significant negative impact on performance.  If that is the case, there are potential solutions
to that problem that are not presented in this revision of the architecture.
}

\lsubsection{Altering Extracted Information}

\newtextPointThree{
There are at least a few situations where the extracted information will need to be
subsequently modified.  Two of those are addressed here.
}

\lsubsubsection{Modification and Overrides}

\newtextPointThree{
This includes Hit Die Locking, Movement Modifiers, and controlling
for a Preferred Class of a Race.  Each of these can be done by using the concept of 
a Modifier.  A Modifier is a form of CDOM Content that can be placed into the PCGenGraph and
subsequently granted to a PlayerCharacter by a PObject.  These Modifiers will take in an
object of a given type and return another object of that type:\\
\texttt{public T applyModifier(T obj);}\\
Thus, this is a general format for Modification that can perform a set, addition, subtraction,
or various other modifications to a given object type.
}

\lsubsubsection{Bonuses}

\newtextPointThree{
Hopefully, these are relatively self-explanatory, but it is possible for many items
to provide a bonus to a particular Skill or other mathematical value within the 
PlayerCharacter.  These BONUS objects will have to be independently acquried from 
the PlayerCharacter's PCGenGraph, tested for validity, and applied if appropriate.
}

\lsubsection{New LST Import Cycle Summary}
\index{LST file}\newtextPointZeroFive{\begin{description}
\item[Import Game Mode Files] is done to import the Game Mode files contained within 
the System directory.  This is only necessary if the Game Mode file required for the 
Campaign to be loaded is not already present in memory.
\item[Import LSTs] is done to import the \ix{LST file}s for the Campaign to be loaded.  This 
is done in a fashion that is very similar to the process in the 5.11 series.  However, 
instead of creating objects and registering them with a Global Class, a \ix{LoadContext} is 
passed to each of the Token\index{token} \index{plugins}plugin Classes, and all operations
are done through the \ix{LoadContext}.  The \ix{LoadContext} manages the building of the
master PCGenGraph that is defined by the contents of the \ix{LST file}s. In this way, more
than one Campaign can be loaded without conflicting in memory. \newtextPointThree{No 
information is destroyed in this step.  Objects which are .MODed will be maintained
within the master PCGenGraph at this stage, as it may be important to resolve the
original source of the information (in order to properly save it back to the
appropriate LST file).  Objets which are MODed, or forgotten or otherwise modified
in that sense will be marked as invalid, but left in the graph.  
It is therefore possible that this early stage will store 
items into an uncached graph. This will improve import speed and later steps can 
construct a cached graph when it is necessary (such as during Prune for Replaced).}
\newtextPointTwo{
\item[Build derivative objects] such as SubClasses is performed in order to ensure
that the entire original object has been constructed before information is derived from
that object (and anything added or removed).
\item[Resolve References] This is the `second pass' of construction, which will
resolve any CDOMReference objects in the graph to ensure that the references in the
master PCGenGraph are actually to real PObjects and not to placeholder objects.  This
may implicitly perform the next step.
}
\item[Build TYPE and ALL/ANY associations] is done to ensure that any references made
by the \ix{LST file}s to Types (such as Skills that are TYPE=Craft) are built within the 
master PCGenGraph.  Using this method, only one centralized reference needs to be 
stored (to the Type or All reference).  These references must be an Allows reference, it 
must not be a contains reference (due to how MONCSKILL will work).
\newtextPointThree{
\item[Prune for Replaced] is performed to remove any references that were replaced.
This eliminates original objects that were .MODed in an LST file.
This step is not performed if the imported LST files will be edited. (Alternately, 
if the editors must stay integrated, then the unpruned graph should probably be 
maintained in memory).
}
\item[Check for invalid references] is performed following \ix{LST file} import.  An equivalent
operation is not performed in the 5.11.* series.  Since all objects are created or 
referenced through the \ix{LoadContext}, this check will ensure that any object that was
referenced in the \ix{LST file}s (such as a Skill granted by a Template) was actually created.
It will be illegal (and will cause an error) to reference an item that is not defined in
the appropriate \ix{LST file}.\footnote{There may be exceptions, such as for Natural Attack
Weapon Proficiencies, which are generated automatically}
\item[Validate Master PCGenGraph] is performed following other actions, in order to ensure that
all of the associated constraints on a Graph are enforced.  For example, there are constraints on
Weapons (a Type of Equipment) which require each Weapon to be either Type Melee or Ranged.
\end{description}
}

\lnsection{Code Structure}

\lsubsection{Basis of the PCGen Token System Architecture}

\lsubsubsection{Background}

\newtextPointFour{
The tokens are a significant investment due to the complexity of parsing LST files.  In
order to reuse this investment across both the LST editor and the runtime PCGen, the tokens
need to be independent of the structure stored internally to the program.  There are 
also other considerations in the token design:
\begin{itemize}
\item A Token unparse method should have limited awareness of .MOD
situations, thus the arguments should only include *one* PObject (not a
PObject copy before and after a .MOD)
\item It would be preferable to avoid implementing clone() (or some equivalent)
in CDOM objects.  This includes for handling .MOD driven changes. (see Section \ref{Minimize PObject Cloning})
\item A token should be as simple as possible
\item A broken token (returns false) should never apply itself
(*especially* partial application) to the graph.  This is to protect the
system from ambiguous, unexplained or undesirable states.
\item Code should be as straightforward as possible.
\item To avoid duplication, the tokens should be reusable
across the runtime program and the PCGen LST file editors.
\end{itemize}
}

\lsubsubsection{Alternative Considerations for Token Architecture}

\newtextPointFour{
A static token (e.g. ``Damage'' in Equipment LST files) is subject to
replacement.  In three files: \\
\texttt{
\noindent Sling	Damage:1d4 \\
\noindent Sling.MOD	Damage:1d4+2 \\
\noindent Sling.MOD	Damage:1d4 \\
}
}

\newtextPointFour{
This could be set in a very simple way: \\
\texttt{
\noindent parse(LoadContext context, Equipment eq, String value) \\
}
calls \\
\texttt{
\noindent eq.set(StringKey.DAMAGE, value); \\
}
...very simple, straightforward, and understandable.
}

\newtextPointFour{
Unfortunately, this doesn't work for the ``editor mode''.  If source files
1 and 3 are loaded (but not 2), then the editor still needs to be aware
that source 3 set the Damage.  If it fails to recognize it was set (to
an identical value, in this case), the editor would (incorrectly) not write out
a .MOD in source three (thus 'accidentally' modifying the source).  This
then introduces an error if source 3 is loaded in production with
sources 1 and 2... (because source 2's .MOD would be the only .MOD that
occurs).  As the Data Monkeys would point out, this is a major data
integrity failure.
}

\newtextPointFour{
Conclusion: the editor cannot use deltas between two CDOM objects
(before and after each .MOD) to track what a source did.  This means
.clone() or some equivalent is really useless to handle the situation.
}

\newtextPointFour{
Alternative A: Store each item with the source it came from, e.g.
put(StringKey.DAMAGE, sourceURI, damageStr)
}

\newtextPointFour{
Alternative B: Instead of cloning, just pass an empty object into a
token when it was called as part of a .MOD and then merge the
lists/arrays after the import process is complete.
}

\newtextPointFour{
\begin{description}
\item[Problem 1 with A] Puts tremendous complexity into the CDOM object,
forcing it to track the source of every object\ldots worthless overhead in
production.
\item[Problem 1 with B] Non-direct tokens like LANGBONUS become very
complicated, as there will be multiple objects for a given Key that
must later be consolidated, which means the graph edges will also have
to be consolidated to the master object.  This also drives a complex
search when .CLEAR. is used in a non-direct token, meaning the solution
seems simple for direct tokens, but adds huge complexity for non-direct
tokens.
\item[Problem 2 with A and B] Doesn't help with .CLEAR, because it would have
to track removal as well.  Therefore the CDOM Object would have to have
a positive and a negative list for every item.  I believe this is an
unreasonable burden on the CDOM object (too much unused stuff in a
production environment - half the object!)
\item[Problem 3 with A and B] Resolving .CLEAR from a series of .CLEAR.
entries difficult.  Given: \\
\texttt{
\noindent Thing	LANGBONUS:English,French \\
\noindent Thing.MOD	LANGBONUS:.CLEAR.English,.CLEAR.French \\
}
Thing.MOD effectively performs a .CLEAR, but one should not alter the
intent, because if source 1 later changes to: \\
\texttt{
\noindent Thing	LANGBONUS:English,French,German \\
}
Then Thing should retain German\ldots so not only is a positive and
negative list required, but some form of identifier to indicate a CLEAR
was (or wasn't) a global .CLEAR
\end{description}
}

\lsubsubsection{Token Architecture Overview}

\newtextPointFour{
Instead of writing directly into any given CDOM object, the Token will
tell the context what is desired.  Interestingly (and accidentally, I
might add), this also abstracts the Tokens from the specific
implementation of the CDOMObject.  e.g. \\
\texttt{
\noindent parse(LoadContext context, Equipment eq, String value) \\
}
calls\\
\texttt{
\noindent context.obj.set(equip, StringKey.DAMAGE, damageStr); \\
}
}

\newtextPointFour{
In an editor environment, this can then be queued up in a Command to
ensure that each source is fully restorable, while in a load for
production environment, the set can be directly applied to the object.
}

\newtextPointFour{
Depending on the type of token, the process for using the Load Context 
to define the relationships will differ.  The following sections describe
how each of the token types will interact with the Load Context and the 
form of the resulting structure in the CDOM Graph.
}

\sbnote{Source URI}{Note
that the sourceURI does not necessarily need to be sent into the
token, as the context can be made aware of it and use it behind the
scenes, if necessary (only in an editor environment).
}

\lsubsubsection{Primary Information Token Architecture}

\newtextPointFour{
For a description of Primary Information Tokens, see Section 
\ref{Primary Information Tokens}.
}

\newtextPointFour{
Primary Information Tokens are generally ``atomic'' in nature, in 
that they are storing relatively simple information directly into the
CDOM Object.  Due to this simplicity, calls to the set* methods
of the Load Context must not be made if the token intends to return
false (to indicate a failed parse).
}

\newtextPointFour{
For each type of characteristic that can be set in a CDOM Object (String,
Integer, Object, List, etc.), an associated set* or add* method exists 
in the LoadContext.  A Primary Information Token should \emph{never} call
a method on CDOM Object.
}

\newtextPointFour{
Primary Information Tokens store information directly into the CDOM 
Object; therefore, they do not impact the CDOM Graph.
}

\lsubsubsection{Association Token Architecture}

\newtextPointFour{
For a description of Association Tokens, see Section \ref{Association Tokens}.
}

\newtextPointFour{
Association Tokens create a relationship between two objects.  Due to potential
complexity in parsing an Association Token for the multiple relationships that 
may exist, calls to the grant() or allow() methods of LoadContext are queued 
for later processing.  The requested associations are built if and only if
the Token returns true.  This allows the token to focus on the relationships
that would be built, rather than queuing data in order to avoid building any
relationships before the token is known to be valid.
}

\newtextPointFour{
When one object grants another object, the relationship stored in the CDOM Graph
will be a direct relationship between the objects.\footnote{This relationship may not be created
at the time the token is executed, but any references used in a grant relationship
will be resolved to the actual object as part of the LST load process.  See Section 
\ref{New LST Import Cycle Summary}}  An example of this is the VFEAT Token:\\
\texttt{\noindent Law <tab> VFEAT:Aura of Law}\\
The resulting CDOM Graph structure for this VFEAT Token (taken from the RSRD Basics Deity
LST file) is shown in Figure \ref{Fig: VFeat}
}

\mapscalefig{scale=0.4}{vfeat.png}{\label{Fig: VFeat}Classes, Relationship and Object contents created
from the use of VFEAT}

\deletedTextPointFive{\newtextPointFour{
When one object allows another object, the relationship stored in the CDOM Graph
must be an indirect relationship between the objects.  The LoadContext will enforce the
requirement that the target of an allows relationship is a CDOMReference, not a 
concrete CDOMObject or PrereqObject.  An example of this is the CSKILL Token, when used 
outside of the Class LST file:\\
\texttt{\noindent Travel <tab> CSKILL:Wilderness Lore}\\
The resulting CDOM Graph structure for this CSKILL Token (taken from the RSRD Basics Deity
LST file) is shown in Figure \ref{Fig: CSkill}
}}

\deletedTextPointFive{
\mapscalefig{scale=0.4}{cskill.png}{\label{Fig: CSkill}Classes, Relationship and Object contents created
from the use of CSKILL (outside of the Class LST file)}
}

\deletedTextPointFive{\newtextPointFour{
The relationship for allows must be indirect and use a CDOMReference in order to control the size
of the Player Character CDOM Graph.  If the allowed object was directly referenced in the PC's CDOM Graph,
then any objects the allowed object grants would also appear in the Graph.  Such a design would produce
additional complexity in resolving the active (granted) objects in the Graph as well as causing 
unnecessary ``theoretical'' information to be written out to a persistent form of the CDOM Graph (the 
save file for a Player Character).
}}

\deletedTextPointFive{\newtextPointFour{
The additional performance cost of performing a reference resolution at runtime is negligible.
}}

\lsubsubsection{List Modification Token Architecture}

\newtextPointFour{
For a description of List Modification Tokens, see Section \ref{List Modification Tokens}.
}

\newtextPointFive{
When one object allows another object, the relationship is stored in a List.  These Lists
are managed separately from the CDOM Graph, but are also resolved, so that Prerequisites can
be used to limit additions to a List.
}

\deletedTextPointFive{\newtextPointFour{
Third Party Tokens create a relationship between three objects.  Due to the
complexity in parsing a Third Party Token for the multiple relationships that exist,
calls to the grantThirdParty() or allowThirdParty() methods of LoadContext are queued 
for later processing.  The requested associations are built if and only if
the Token returns true.  This allows the token to focus on the relationships
that would be built, rather than queuing data in order to avoid building any
relationships before the token is known to be valid.
}}

\deletedTextPointFive{\newtextPointFour{
As with the basic grant relationship, when one object grants another object, the
relationship stored in the CDOM Graph will be a direct relationship between the granting
object and the granted object.\footnote{Again, this relationship may not be created
at the time the token is executed, but any references used in a grant relationship
will be resolved to the actual object as part of the LST load process.  See Section 
\ref{New LST Import Cycle Summary}}  A contains relationship is also maintained to 
the object which is granting the relationship.  An example of this is the LANGBONUS Token:\\
\texttt{\noindent Halfling <tab> LANGBONUS:Dwarven}\\
The resulting CDOM Graph structure for this LANGBONUS Token (taken from the RSRD Basics Race
LST file) is shown in Figure \ref{Fig: LangBonus}
}}

\deletedTextPointFive{
\mapscalefig{scale=0.4}{langbonus.png}{\label{Fig: LangBonus}Classes, Relationship and Object
contents created from the use of LANGBONUS}
}

\lsubsubsection{Replacement Token Architecture}

\newtextPointFour{
For a description of Replacement Tokens, see Section \ref{Replacement Tokens}.
}

\newtextPointFour{
At this time, an exmaple is not provided.
}

\lsubsubsection{Factory Token Architecture}

\newtextPointFour{
For a description of Factory Tokens, see Section \ref{Factory Tokens}.
}

\newtextPointFour{
At this time, Factory Tokens are not well defined.
}

\lnsection{Transition from PCGen 5.12}

\lsubsection{Prerequisites for Transition}

\newtextPointZeroFive{
In order to perform some of the operations that are proposed in this document, 
certain changes must be complete in the 5.11.* cycle in order to make
a clean transition possible
\begin{itemize}
\item Fix: PRERACETYPE is not a standard PRE tag, needs to follow the standards
( \url{http://tech.groups.yahoo.com/group/pcgen_experimental/message/6281} )
\item Fix: ADD and CHOOSE should be modified to require a sub-token
(e.g. ADD:$<$subtoken$>$\vbar$<$stuff$>$)
( \url{http://tech.groups.yahoo.com/group/pcgen_experimental/message/6683} )
\item Eliminate CHOOSE:LANGAUTO: special case in Race Loader
\end{itemize}
}

\lsubsection{Required Steps in Transition}

\newtextPointZeroFive{
There is no reason for the conversion to CDOM to be a major cut-over.  Here is the general
list of steps to incrementally convert to a full CDOM environment:
\begin{itemize}
\item Establish Library Classes (including Graph Classes)
\item Convert Plugins to load the current structure and the PCGenGraph structure
\item Convert Plugins to have an unparse method
\item Convert the PlayerCharacter reader/writer to support both the current core
and the CDOM structure
\item Convert each query of the core (export token, GUI field, etc.) individually
to use the PlayerCharacter PCGenGraph rather than the existing PObject-based core.
(start with pulling languages possessed and
available from the Graph and remove it from the old core - then move on
to Spells or Skills)
\item Convert the LST editor to be a PCC/LST editor, using the CDOM Graph structure and
the token unparse() method
\item Create a new PlayerCharacter reader/writer that is completely based on the
CDOM structure (.xpcg).
\end{itemize}
}

\lsubsection{Transition Detailed Steps (Not Complete)}

\lsubsubsection{Reserved for 5.13.*}

\newtextPointTwo{
\begin{itemize}
\item Place base library classes in SVN (done in CDOM branch)
\item Place graph library classes in SVN (done in CDOM branch)
\item Place CDOM classes in SVN (done in CDOM branch)
\item Convert other PrereqObject references to ConcretePrereqObject (done in CDOM branch)
\item Inline ConcretePrereqObject facade methods (done in CDOM branch)
\item Delete Deprecated Tokens (important, since those will not be supported) (done in CDOM branch)
\item Convert PObject to extend CDOMObject (done in CDOM branch)
\item Alter GlobalLstToken to add LoadContext as first Arg (done in CDOM branch)
   (Use Change Method Signature in Eclipse!)
\item Duplicate LstObjectFileLoader's parseLine and add LoadContext as first Arg
   (Or add ParseToken if Loaders have been Generalized)
   (Use Change Method Signature in Eclipse!)
\item Alter GlobalLstToken to require a method without a level (for future use)
\item Make Movement a PrereqObject
\item Alter Loaders to properly handle a LoadContext
\item Add Graph/LoadContext-based function to Global Tokens
   (Requires some mods to Content objects like Vision, DamageReduction)
\item Convert Token interfaces and tokens to add graph-based functionality
\item Start the hard work of copying graph components into the PC and
extracting information from the graph :)
\end{itemize}
}

\lsubsubsection{Reserved for 5.15.*}

\newtextPointTwo{
\begin{itemize}
\item Generalize Loaders so that they are not repeating work and code
\item Move ModeTokens to plugin.modetokens.*
\item Move SponsorTokens to plugin.sponsortokens.*
\item Move LstToken Interfaces to pcgen.persistence.lst.token
\item Make PCClassLoader no longer extend LstObjectFileLoader and remove PCClass specific function
from LstObjectFileLoader
\end{itemize}
}

\lsubsubsection{Specific Fix for PCClassLevel}

\newtextPointTwo{
\begin{itemize}
\item  Need to handle in a special way the level-based processing of PCClass
\item  Need to handle in a special processing of SubClass and SubstitutionClass
\end{itemize}
}

\lsubsubsection{Clean Up}

\newtextPointTwo{
\begin{itemize}
\item Remove Level awareness from All Tokens (especially Global Tokens)
\item Convert util.enumeration refernces to cdom.enumeration
\end{itemize}
}
 
\lnsection{PlayerCharacter cases and architecture refinement of the PCGen CDOM}

While the high level explanation above provides a general explanation of how information will
be stored after it is imported into PCGen and when added to a PlayerCharacter, there are some
specific cases that should be addressed in order to clarify how they fit into the proposed
CDOM model.

\lsubsection{Choices}
	
Example: TEMPLATE:CHOOSE: tag

This requires that a ChooseList object be inserted into the PCGenGraph.  ChooseList will be a
separate object (derived from \ix{PrereqObject}).  A Choose without a
selection can be highlighted to the user as an unselected option on the PlayerCharacter. 
Reselection also becomes easy, as the ChoiceList will still be `aware' of the legal choices.

\newtextPointThree{
All Choices will be hung off of a ChoiceList object, which enforces the
rules for what it can be connected to (the ChoiceList will know it does
Skill objects, and the allowed skills are X, Y, and all of Type Z).
Therefore, any choice could be retroactively changed.
}

\sbeffect{Operation of Reselecting}{\newtextPointThree{Reselection
capability means that there will need to be some ability to highlight
when a character has violated the rules without stopping the user
action. (If you swap out a Feat that was a Prereq for a later Feat, you
don't want to prevent the action - that makes changes too complex
because it forces things in a reverse order - but you must flag the PC
as violating the rules... so put up a red exclamation point or something)
}}

\sbeffect{Object Association}{\newtextPointThree{This may not be clear
from other locations in this document, so there will be a requirement for the 
GUI to provide a view for PCs that can allow object associations - especially
choices - to be understood (what came from where) as
well as to be changed, either by dragging them to another location (to
fix a prereq violation) or by replacement.
}}

\newtextPointZeroFour{
The major problem with CHOOSE tags throughout the PCGen system is their format inconsistency. 
In order to properly parse these in the \ix{LST file} import system, it is proposed that the format of
ALL of the CHOOSE \index{token}Tokens be changed in 5.12 (for removal of the old format before 6.0).  This
`new' format for the CHOOSE Tokens will be: CHOOSE:$<$CHOOSETYPE$>$\vbar$<$STUFF$>$  I would propose that
the CHOOSE Token actually drives a separate \ix{LST file} \index{loader}Loader
to pass in the CHOOSETYPE as the Token
Key and the STUFF as the value.  This actually separates the different CHOOSE values into 
separate Tokens, as I think this will be valuable for readibility, as well as controlling
the size of the ChooseLstToken (which would otherwise become a HUGE IF statement with tons 
of different permutations for how to parse the tokens). 
}

\newtextPointZeroFour{
As a side note (while I'm discussing the \index{loader}Loaders), I think it's valuable to have ADD operate
in the same fashion (of a separate SUBLOADER).  This allows more flexibility, while controlling
Class size.
}

\newtextPointZeroFour{
In addition, Template's HD this could easily share \index{token}subtokens with LEVEL and REPEATLEVEL,
so this requires a little bit of thinking.  Generally, since those three
Tokens share a bunch of analysis code (and their targets are similar), it would really be nice
to be eliminate that duplicate code, this might be doable with another \ix{loader}???
}

\newtextPointZeroFour{
I am unsure whether this separate \index{loader}Loader philosophy works well for
\index{bonus}BONUSes... haven't spent as much time on those.
}

\lsubsection{Clearing tags with .CLEAR}

Since many of the tags with the \ix{LST file}s can be cleared or otherwise have objects removed
under certain circumstances, the entire PCGenGraph must somehow be available to any tag during
LST load.  This is done through the \ix{LoadContext}.

\newtextPointZeroFour{
Note also that each \index{token}token will be limited to clearing only those objects placed into the master
PCGenGraph by that token (this can be done as the source Token for each PObject reference was 
stored).\footnote{This will not be a strict enforcement, as it is postulated that a draconian 
enforcement of this is impractical and degrades the ability to deprecate objects and allow
updated operation.  This is probably better phrased as a default behavior rather than a limitation,
but suffice it to say, violating this assumption should only be done with a very good cause.}
}

\lsubsection{Association based prerequisites}

\revisedtextPointZeroFive{
Example: TEMPLATE:Foo$<$tab$>$DOMAIN:Sun[PREDEITY:SunGod]
}

\revisedtextPointZeroFive{
Many objects only grant a character other objects under specific
circumstances. These are generally represented by having PREREQs on the
granted object. The example above means that the Template Foo will grant
the character the Sun Domain, but only if they have the Deity SunGod.
As this impacts the Template, but should not alter the Domain,} this
situation requires the restriction be kept separate of the ``contained''
PObject (since the child PObject is not being restricted in all cases, and must not be \ix{clone}d
[Assumption \ref{Minimize PObject Cloning}]).  Since this \ix{Prerequisite} is part of the association,
placing it in the graph
edge is a good mental model.  Therefore, each PCGenGraphEdge will be able to contain Prerequisites
in order to prevent `traversal' of the graph to PObjects that are prohibited by their 
(association) Prerequisites.

\lsubsection{Group References}

\revisedtextPointZeroFive{
Example: CSKILL:TYPE=Craft
}

Some of the tags within PCGen allow for all objects of a given type to be referenced all at once. 
It would be preferable to resolve the type only once (during \ix{LST file} import) and not every time the
given information needs to be processed.  Therefore, the \ix{LoadContext} is also capable of returning
a reference to a TYPE:\\
\revisedtextPointZeroFive{
\noindent\texttt{//Type is ``Craft'' in token example above\\}
\noindent\texttt{String type = tok.nextToken(); \\}
\noindent\texttt{craftskilltype = context.getTypeRef(SKILL\_{}CLASS, type);\\}
}

After this reference is returned, and at the completion of the \ix{LST file} import, the \ix{LoadContext} can
go back into the master PCGenGraph and build edges connecting this master Type reference to all
of the objects in the master PCGenGraph of the given Class that are of the appropriate type. 
References to ``ALL'' or ``ANY'' can be resolved in similar fashion through
\ix{LoadContext}'s \texttt{getAllRef} method.

\lsubsection{Object Addition on Other Object Insertion}

Example: Template's REPEATLEVEL tag

\revisedtextPointZeroThree{
This type of tag may force the pre-building of PCLevel objects.  This may cause a bit of memory
usage in cases where those Classes are not used, but is recognized as a reasonable trade-off to
simplify code in the first pass of CDOM.  A more complicated structure is possible if performance
becomes an issue due to memory use or excessive prerequisite testing driven by this (and similar) 
Tags.
}

\newtextPointThree{
This token is also complicated and therefore interacts with editors.  It would be beneficial 
with a tag of this complexity to be able to `recompress' the information back down to a
REPEATLEVEL Token in the case where the master PCGenGraph is written out to LST files.  
Therefore, this token will use an Aggregator node in the PCGenGraph in order to store
the information in the REPEATLEVEL to allow it to be recompressed into one Token.
}

\lsubsection{Temporary Unqualified objects}

In some cases, a PlayerCharacter will make a choice (of a Feat, for example) that may have
\ix{Prerequisite}s that the PlayerCharacter later becomes ineligible for.   
\revisedtextPointZeroFour{
Currently, no special 
processing is required for this case, as the PlayerCharacter graph will contain other objects the
PC is ineligible for, and a graph traverse will be performed on each query of the PC.
}

\lsubsection{Equipment / Character Association}

\revisedtextPointTwo{
In order to be able to compose complex equipment, as well as ensure that containers are 
properly handled,\footnote{The threads discussing the background of this issue
(entering at a random point) are available here:
\url{http://tech.groups.yahoo.com/group/pcgen_experimental/message/5627}
\url{http://tech.groups.yahoo.com/group/pcgen_experimental/message/5227}
} there are multiple associations that are required for equipment.  
}

\revisedtextPointFive{
First, in order to ensure that equipment that is carried by a PC is not considered to be
equipped by that PC, there will be a separate Equipment List identifying what Equipment the PC carries.
(This allows the PC to possess equipment, but not actually have that equipment impact the Player Character's attributes.)
Second, once equipped, any equipment should impact the PC.  At that point, the Equipment is placed
into the PlayerCharacter Graph (it is granted to the PC).
}

\revisedtextPointTwo{
A complex piece of Equipment could have EQSLOTs, which would allow for other equipment
types to be contained within the Equipment.  (The PlayerCharacter Sockets are defined by 
the Game Mode, in the Equipment Slots file).  Anything placed into an Equipment Slot is 
considered ``equipped'' by the Player Character, or an active component of a larger object
if the parent object is also equipment.
}

\revisedtextPointTwo{
A piece of equipment could be modified to have further EQSLOTs by adding an EqMod that provides
additional slots (This means that for pre-defined construction, Equipment directly contains EQSLOTs 
which directly contain other equipment, while later add-ons such as the Modern multi-weapon stuff
would be Equipment that contains an EqMod that contains an EqSlot that contains more Equipment).
}

\revisedtextPointTwo{
EQSLOTs can be restricted, which would only allow certain types of equipment (or exactly
certain equipment) to fit into an EqSlot.  This could also be done with EqMods, but given the
distinction between EqSlots and and EqMods made here, I do not believe that is necessary at this
point.
}

\revisedtextPointTwo{
Pockets also provide a challenge.  I believe there should be an EQHOLDs Token that provides this
capability (this may also mean an EqHolder object). (A Player Character is
effectively granted EQHOLDS:ANY\vbar Infinity, since it's really weight that limits the PC).  
This allows Equipment to hold other equipment (e.g. pockets, patches, etc.)
without having that equipment activated when the parent equipment is equipped by a Player Character.
}

\lsubsection{Equipment / Alternate Heads}

\newtextPointZeroFour{
One challenge with Equipment is there exists the possibility of having a two-headed weapon
(such as a double axe).  This provides a challenge, considering that Equipment Modifiers are 
independent CDOM objects and therefore neither the Equipment nor the Equipment Modifier (EqMod)
is aware of the other.
}

\newtextPointZeroFour{
The actual problem lies in the fact that an EqMod applied to a piece of Equipment only impacts
one of the two heads (there is an EQMOD \index{token}token and an ALTEQMOD token).  CDOM cannot blindly 
attach both to the Equipment as the parent, or any ability to resolve the actual `owner' of
the EQMOD is lost.
}

\newtextPointZeroFour{
It is therefore proposed that every Equipment object will receive an EquipmentHead object
attached to it that is defined as the Primary head (I suggest we number these, on the off
chance that we actually come up with something that needs more than two).  If any of the ALT*
Tokens\index{token} are encountered, then the object will have a secondary head attached to the 
Equipment. The ALT* tokens are: ALTCRITRANGE, ALTDAMAGE, ALTEQMOD, ALTTYPE.  Information from
these tags (and their non-ALT* equivalents) will always be placed into the EquipmentHead, not
the Equipment itself.
}

\newtextPointZeroFour{
As a result of this, some function must be provided in the \ix{LoadContext} to extract information
back out of the PCGenGraph (e.g. \texttt{getChildrenOfNode(PObject, Class)}).  This will 
allow the Heads to be extracted back from the Graph to get the PObject to add the information
to, as well as given the Token\index{token} the ability to test whether an alternate head exists (and if not,
to add it to the master PCGenGraph).
}

\lsubsection{Associations}

\newtextPointZeroFour{
\revisedtextPointZeroFive{
When associations are made that are related to a specific reference to a PrereqObject (this may be
hit points, a subcategory of a Feat, or cost of an Equipment Modifier), that information
will be stored as an attribute of the edge serving as a source to that \ix{PrereqObject}.}
Therefore, an EQMOD, for example, of 
\makebox{MAGIC\_{}COST\vbar{}9500} has the 9500 stored in the edge that 
connects the EqMod to the EquipmentHead to which the EqMod was applied.
}

\lsubsection{Skill Levels}

\newtextPointZeroFour{
The level of a Skill will be defined by the number and weight of the edges connected to the Skill
within the PlayerCharacter's PCGenGraph.
}

\newtextPointZeroFour{
For example, if a PC has a Hide Skill of 4, and Hide is a Class Skill, there will be 4 incoming
(source) edges connected to the Hide Skill.  If two of the edges are classified as cross-class 
edges (because the PC is multi-classed), then the Skill value will be 3. 
}

\newtextPointZeroFour{
It is possible that the edges connected 
to Skills in this fashion will be categorized as class edges, cross class edges and exclusive edges.
In this way, the weight of each of these can be altered independently of the associations within 
the PCGenGraph.  \revisedtextPointZeroFive{However,
altering the associated cost of class/cross-class/exclusive Skills introduces
an unresolved complexity, due to issues with order of operations (if a template added at level 2, does it
impact Skills taken at level 1)}.\footnote{The unresolved 
case was originally highlighted
here: \url{http://tech.groups.yahoo.com/group/pcgen_experimental/message/6305}}
}

\lsubsection{Stackable and Multi-Selection Feats}

\newtextPointZeroFour{
Stackable Feats will operate similar to Skill Levels, in that the number of connections to the Feat
indicates the number of times the feat is available (and thus applied).  Note that each path will 
have to be checked back to the PlayerCharacter root to ensure that all intervening \ix{Prerequisite}s 
are actually met.
}

\newtextPointZeroFour{
Multi-Selection Feats (those that can be selected multiple times with different associations) will
operate in a similar fashion, but the incoming edges will define the association.
}

\lsubsection{Hit Die}

\revisedtextPointZeroFour {
Given the existence of the HitDieLock items, and the desire to use the
association holding of an Edge, it would seem valuable to have HitDie 
and HitDieLock deserve primary treatment (e.g. a PrereqObject that appears in the
master PCGenGraph) equal to other content (like DR, SR, etc.).  The assocation will contain
the number of hit points assigned by the Hit Die.
}

\newtextPointZeroFour {
Since the Hit Die is a static object that will be present in the master PCGenGraph, there
will need to be a mechanism for determining when an edge requires an association to be made
before it is imported into the PlayerCharacter PCGenGraph.  This model of having an edge that
requires an association will also be available for those PObjects which have other associations
(e.g. Weapon Focus).
}

\lsubsection{Object Addition on Self Insertion}

Example: Template's ADDLEVEL tag

Some PObjects trigger the addition of advanced characteristics when that given PObject is added
to a Player Character.  An example of this is Template's ADDLEVEL tag.  This triggers adding
character levels of a certain PCClass.  This will be a factory object.  The factory will be placed
in (or perhaps connected to?) the Template, and triggered when the Template is added to the
Player Character.

This factory is seen as preferable to pre-building PCLevel objects and adding the appropriate
PCClassLevel objects with Prerequisite edges based on the given Template.  That would highly
confuse the master PCGenGraph, require significant coding to ensure it was correct, and also
negatively impact performance of any character that does not contain the Template that drove
that edge insertion.
\newtextPointZeroFour{
Actually, it's pretty much impossible to create a static link to the PCLevels
because the Template must
not assume it knows the existing level of the PC.
}

\deletedTextPointZeroFour{
\revisedtextPointZeroThree{
I'm actually not 100\% sure about this any more; I'm still struggling with the
PCClassLevel thing (section \ref{Avoiding a Contract with PCLevel and PCClassLevel})
and that will reflect on this (given that it originated from an ADDLEVEL tag). 
That's not to say there were not other potential reasons for a Factory, that will require
further investigation\ldots
}
}

\revisedtextPointZeroThree{
\deletedTextPointZeroFour{If this does happen, then it}
\newtextPointZeroFour{This Factory} must fetch the instance \revisedtextPointZeroFive{of the
added object} from the master PCGenGraph, as the instance MUST exist there
\revisedtextPointZeroFive{(referencing an object which is not defined in the \ix{LST file}s
is always an error).}
}

\newtextPointZeroFour{
Generally, the desire to make the references from a template to a PCClassLevels be static
is a problem.  While ideal to make the master PCGenGraph entirely static, the template
cannot assume it knows the PCLevel, so Feats or other items that are assigned based on
the PCLevel would never be properly applied without a factory and deferred instantiation.
}

\lsubsection{Saving Characters}

\index{save}
\revisedtextPointZeroThree{
We should be able to save all the data that
was used to construct a character out with said character so that it should be easy to ensure in
the future that said PC can always be faithfully reloaded.  This probably has other consequences,
such as the potential need to allow full separate graphs to be loaded at the same time, even where
there are name conflicts.  The general case is that you should be able to load a character, and
have it reflect the most up to date data loaded, but you should also be able to load the character
to a last known working graph.  This would assist both the programmers, and the lst monkies.
}

\newtextPointZeroFour{
Actually, I assume here that ``all the data used to construct a character'' is not the entire
campaign but only what the PC selected?  If this is only what the PC selected, then we should look
at simply serializing the Player Character PCGenGraph out in XML.
}

\lsubsection{Avoiding a Contract with PCLevel and PCClassLevel}

It could be noted in previous use cases that I am referring to both a PCLevel and a PCClassLevel. 
These are associated (in that they should be added or removed together from a PlayerCharacter),
yet also independent objects.  \newtextPointZeroTwo{However, this is not
definite, as there are multiple methods of handling levels.}

\deletedTextPointZeroFour{
One proposal would make link from the PlayerCharacter
to the PCLevel and PCClassLevel be a 1 source (the PCRoot), 2 sink (PCLevel and PCClassLevel)
edge.  By the same logic discussed in the detailed evaluation of
Case \ref{Object Addition on Other Object Insertion}, this will mean that
removal of the edge or either node will result in both the PCClass and PCClassLevel being
removed at the same time.  Replacing a PCClassLevel with a different PCClass' Level could be performed
by using the edge replacement method discussed in case \ref{Slots}.
}

\newtextPointZeroTwo {
One challenge in fully separating the PCClassLevel from PCClass 
is the difficulty in handling Spell casting ability (CAST, KNOWN, etc.).  The problem is that some
magical items, feats, or other characteristics of a PlayerCharacter are capable of providing
a \ix{bonus} to the Effective Caster Level.  The challenge is that if a PCClassLevel is only aware 
of the Spell characteristics for that Level, it will need to query another object for the 
information for the alternate level.  This effective shifting of caster level removes some of
the incentive for separating (at least) the Spell casting information related to PCClassLevel 
from the PCClass.
}

\deletedTextPointZeroFour{
\newtextPointZeroTwo {
One solution to this would be for the SpellProgression to be duplicated 
through each of the PCClassLevel objects, and have the PlayerCharacter only query the 
highest level PCClassLevel object for any Spell characteristics.
}}

\deletedTextPointZeroFour{
\newtextPointZeroTwo {
Another possibility would be to skip creating a PCClassLevel object at all.  One could
make level implicit in the graph by having multiple 
edges connecting to a PCClass.  
\newtextPointZeroFour {
This makes PCClass share `value' characteristics with Skills and Feats, which helps
to minimize the special casing of PCClass and keeps similar mental models in place.
}
The number of connections (note the edges would also be connected
to a PCLevel object) would indicate the level in that particular PCClass.  This would make 
determining the level a matter of edge counting, rather than searching the PCClassLevel objects
and finding the highest PCClassLevel.  It is postulated that ascertaining the level for a particular
PCClass is approximately equivalent in speed to comparing all of the PCClassLevel objects to
determine the highest level for a given PCClass.
}
}

\newtextPointZeroFour{
It seems the SpellProgressionLevel must be an independent PObject since there are items associated
with it (memorization slots, etc.).
This could introduce some additional complexity that would have to be carefully managed: If a \ix{bonus} 
SpellProgressionLevel is provided by a non-Class, that bonus SpellProgressionLevel must be monitored
in order to make sure that an additional Level of a PCClass which grants a subsequent SpellProgressionLevel 
actually becomes the grantor of the next SpellProgressionLevel in sequence after those assigned by
other PCClassLevels.  (If a 2nd Level Cleric has a bonus Cleric SpellProgressionLevel (3), then when 
the character achives 3rd Level, the Cleric SpellProgressionLevel (3) should be moved in the PCGenGraph
to be attached to PCClassLevel Cleric (3) and the bonus should be `given' Cleric SpellProgressionLevel (4)).
}

\newtextPointZeroFour{
In order to avoid moving objects in the PCGenGraph, I would propose 
an ``artificial focus'' for the SpellProgressionLevels, in that they are 
summarized for a PlayerCharacter.  In this idea, a PCClassLevel links to a SpellProgressionLevelGroup, and
that `parent' then creates a child SpellProgressionLevel for each object which
attaches to the SpellProgressionLevel in the 
PCGenGraph.  (same number of PCGenGraph edges in and out).
This then avoids any reassignment (this is preferred, since it avoids complexity), and
universalizes how SpellProgressionLevels are assigned.  It also would allow additional SpellProgressionLevels to be `saved'
(but be inactive) in case the \ix{bonus} SpellProgressionLevels are provided by equipment which becomes inoperative
or is not currently equipped by the PlayerCharacter.
}

\revisedtextPointZeroFour{
This design would also break out the level dependent items into a PCClassLevel while leaving the
class-based items in the PCClass.  Both could be added to the PCGenGraph.  Thus, when a Player
Character achived a level, it would have a PCLevel object granting certain slots... and also grants
a PCClassLevel object, which grants certain slots, as well as a PCClass (and possibly a SpellProgressionLevel).
In this way, the PCClassLevel can be reselected, if a character is changed, and done 
with minimal pain.
}

\revisedtextPointZeroFour{
Bonuses\index{bonus} are tricky.  One can view them as \textbf{either} a series of +1s that are given at
certain levels, or as a \ix{Formula}.  The series of +1s are more naturally homed with a PCClassLevel,
wherease a \ix{Formula} is more naturally homed with the PCClass.
}

\revisedtextPointZeroFour{
There is a \ix{contract} risk in this \deletedTextPointZeroFour{last} case:
It is possible in a pure Graph to remove the PCClass
or the PCClassLevel without removing the PCLevel - thus leaving the PCGenGraph in an indeterminate
state.  There are a number of ways around this:
\begin{itemize}
\item There could be a ``heavy edge'' that automatically deletes the parent edge if the child node
is deleted (the parent is PCClassLevel in the case of the edge linking to PCClass).  This still may 
leave a light form of contract to properly insert these types of edges in the system whenever 
such inconsistent states could be generated.  I believe this is reasonable.
\item There could be ``subgraphs'' inserted into the PCGenGraph that constitute together multiple
objects.  This gets very difficult to separate out objects that may be contained within more than 
one subgraph (such as the PCClass object) and thus is subject to significant additional complexity.
This is not a preferred option.
\end{itemize}
}

\revisedtextPointZeroFour{
Somewhere in this is a tradeoff between a \ix{contract} risk and complexity.  While an elegant solution 
would be nice, that may not be possible.  The best tradeoff in terms of implementation complexity is
probably to refrain from complex edges (those with multiple sinks - that proposal is not visible in this
document revision) and to have a form of contract on the 
developers to know when certain items are strongly associated and must be removed together.
}

\newtextPointZeroFour{
As an additional note, this discussion implicitly
makes the point that there may be items which require validation in the
PCGenGraph, and we may want a system that is capable of automatically validating any given Player
Character Graph.
}

\lsubsection{Slots}

When a character advances to a certain level, they receive specific Skills, Feats, or other items. 
\deletedTextPointZeroFour{
\newtextPointZeroFour{One option is that} a slot is an edge that DOES NOT possess a Sink. 
In this way, slots are easily identified as
unfilled, can be stored within the master PCGenGraph (if they are not manufactured by factories
of the form in cases \ref{Object Addition on Self Insertion} and
\ref{Object Addition on Other Object Insertion}).  When the slot is filled, 
the edge can be replaced by another
edge that connects to the given PObject.  Replacement is significant, as replacement (vs.
modification in place) of existing edges would require reconstruction of all edges as PObjects
are copied from the master PCGenGraph to the PlayerCharacter PCGenGraph.  Each edge, as defined
by the Interface, will be required to be capable of building a replacement edge of the same type. 
For speed and memory, we will likely be storing concrete edges of different types (one optimized
type for the simple 1 source 1 sink edge, one hyperedge); thus the replacement requirement vs.
an assumption of using a specific Class and new.
}

\newtextPointZeroFour{
Slots should be stand-alone \ix{PrereqObject}s (similar to
ChoiceLists), that simply know what type of object they are allowed to ``acquire'' as a child.
This avoids the weird situation of having edges that are not one source and one sink, thus
making the graphs cleaner.
}

\revisedtextPointZeroThree{
\deletedTextPointZeroFour{
These slots could be attached to BOTH the PCLevel and the PCClassLevel object connected to the
PlayerCharacter.  This is required in order to associate them to the PCLevel (from which they
are received) and the PCClassLevel (which constrains what is a Class Skill and what is not). 
\revisedtextPointZeroFour{
However, making this multiple-association significantly complicates the PCGenGraph, by creating
edges with more than one source or sink.  }
}
\revisedtextPointZeroFour{
In order to avoid multiple-association of the Slot, we should try to 
find a model that allows slots to be stored in the PCGenGraph and still be properly 
associated back to the PObject(s) that constrain Class Skills, cross-Class Skills and exclusive 
Skills.  It is desired to have as little association intelligence in the code as possible. 
(Skills should not be treated as some major exception to processing the Player Character).
This is possible given the capabilities that must surround PRECONTEXT (theoretical edges would
only be traversed if an ancestor of the theoretical edge was part of the current context).
}
}

\lsubsection{Supporting Choices}

\revisedtextPointTwo{
One feature that is not currently supported in PCGen that is
needed is the ability to reference a particular choice from within LST code.  For example, Kits
require the ability to select the value of choices (like Favored Enemy).  This is difficult
(actually I think impossible) to achieve in the current architecture.
}

\revisedtextPointTwo{
Looking at the example of Favored Enemy, we find that is present for a Ranger in the SRD.
This is granted as an Automatic Feat (or an Automatic Ability if you want to get techincal 
about the code).  Looking at the CHOOSE in the Favored Enemy I feat, it will have to be
changed to a 5.12 compatible CHOOSE\footnote{see
\url{http://tech.groups.yahoo.com/group/pcgen_experimental/message/6683}}.  Thus it will appear as:\\
\texttt{CHOOSE:NUMCHOICES|1|STRING|Aberration|...}\\
This will then create a ChoiceList of type String attached to the Feat.
This ChoiceList is available in the Graph, and could be extracted, if a syntax was developed.
One proposal for setting a CHOOSE would be something of the form:\\
\texttt{SETCHOOSE:ABILITY|CATEGORY|Favored Enemy I|Foo}\\
The ``ABILITY\vbar CATEGORY\vbar Favored Enemy I'' gives enough information to uniquely identify
the object in either the master PCGenGraph or the PC PCGenGraph.  The last argument is
the choice that would be made.
}

\revisedtextPointTwo{
Note that an implied requirement here, based on validating inputs, would be that
the SET would have to be calling a real object and setting a valid value.
}

\lsubsection{Reference Resolution after Construction}

\newtextPointTwo{
This section replaces ``Reference Before Construction'' from version 0.1 of this document
}

It will likely be difficult to impossible to create a situation where we can definitively assign
the \ix{LST file}s into a directed acyclic graph in order to always load objects before they are
referenced.

\newtextPointTwo{
However, a ``reference before construction'' (RBC) system is very difficult to maintain without
negatively impacting the internal data structure of PCGen.  Consider that a Feat may grant
another Feat that exists later in an LST file.  This forward reference is possible in a RBC system,
IF AND ONLY IF all object names are Keys.  Given the existing structure of PCGen, this is not
a reasonable limitation.  The specific challenge exists in the following sequence:
\begin{itemize}
\item Discover a reference to Feat Foo, ask for a copy of Foo (a Feat Foo is returned)
\item Discover the construction of Feat Bar, ask for Bar to be constructed.  Bar has not yet
been referenced, therefore a new Feat is returned.
\item Strike the KEY Token of Feat Bar, realizing that the key is actually Foo.
\end{itemize}
The problem in an RBC system is that there are now TWO Feat objects that must be reconciled. 
This could be done by making the Foo copy actually delegate to the real Foo object (constructed as
Bar), but this makes the data model ugly.
}

\newtextPointTwo{
This therefore requires a two-pass load system.  However, two passes on the LST files are not 
necessary, and are actually considered to be detrimental to performance and code clarity.  Consider 
in a slightly different context a Case similar to that above:
\begin{itemize}
\item In the discover pass, discover the construction of Feat Bar, with KEY Foo.  Feat keyed Foo
is constructed and given the Name Bar.
\item In the discover pass, discover the construction of Feat Bar, with KEY BarKey.  Feat keyed
BarKey is constructed and given the Name Bar.
\item In the populate pass, strike an object with the name Bar.  One must then perform a LOOKAHEAD
to discover the KEY tag to discover which of the Feat objects with the name Bar is being constructed.
However, consider the possibility that the KEY token is actually on a .MOD line.  This then 
requires that the lookahead understand .COPY and .MOD.  This is unreasonably complex.
\end{itemize}
}

\newtextPointTwo{
The actual proposal is a resolved reference system.
\begin{itemize}
\item In the construction pass, discover the construction of Feat Bar, with KEY Foo.  Feat keyed Foo
is constructed and given the Name Bar.  All other tokens are processed at this time.  Any references 
(such as granting class skills through CSKILL) are deferred by placing a CDOMReference into the 
master PCGenGraph.
\item In the discover pass, discover the construction of Feat Bar, with KEY BarKey.  Feat keyed
BarKey is constructed and given the Name Bar.  \ldots
\item In the reference resolution pass, replace all CDOMReference objects in the master PCGenGraph
with the actual objects.  This means that only one pass on the LST files is required.
\end{itemize}
There may be some slight complexity in cases where the references are not direct edges within
the Graph.  Therefore, the CDOMReference objects must still allow resolution during PC use
of the Graph (ChoiceLists come to mind here).
}

The reference and construction system operates as follows:
A PObject can be referenced at any time by asking a \ix{LoadContext} for a reference to the object:

\noindent\texttt{wpref = context.getRef(WEAPONPROF\_{}CLASS, tok.nextToken());}

\noindent During the actual construction of the object, a construction method is called:

\noindent\texttt{wpref = context.construct(WEAPONPROF\_{}CLASS, tok.nextToken());}

By managing construction and references it is possible to reference before construction, ensure
that all referenced objects are eventually constructed, and (for any given name) each object is
only constructed once.
At the end of each import cycle, the PObject Classes will be checked to ensure that each reference 
requested during \ix{LST file} import was actually constructed.  If not, then an error will be generated.

Side effect: This architecture follows Tom's preference to have a concrete Class and a read only
Interface to that Class.  This allows a single-pass load and makes object construction very
straightforward, preventing an intermediate factory from queuing up all of the known information
and then regurgitating that information into an object all at once in a \ix{constructor}.

\lsubsection{SubClass and SubstitutionClass}

\newtextPointTwo{
SubClass and SubstitutionClasses will be independent PObjects within the master PCGenGraph and
will likely need to be constructed from a Factory.  The reason for this is that the SubClass and
SubstitutionClass are copies of a PCClass, but they must be allowed to both add and remove
items from the PCClass (if this were solely an addition problem, then they could grant the 
PCClass, but that is not the case).
}

\newtextPointTwo{
Therefore, a derivative object construction pass will be required in order to properly construct
SubClass and SubstitutionClass.  This derivative object construction pass must take place after 
all .MODs, etc., but before any reference resolution takes place.
}

\newtextPointTwo{
Knowledge of the existance of derivative objects must rest with the code that allows objects
to be granted out of the master PCGenGraph into a PlayerCharacter.  Therefore, the code to 
add a level needs to be aware that the PCClass object will have \texttt{getSubClassList()}
and similar methods to resolve the actual object to be placed into the PlayerCharacter Graph.
}

\newtextPointTwo{
It is recognized that this resolution must also take into account the (il)legality of taking
a given PCClass at level one when a SubClass was previously taken (you cannot attempt to take
a second SubClass).
}

\lsubsection{Player Character Active Object Resolution}

\newtextPointZeroFour{
The challenge lies in how a PlayerCharacter is stored.  Since an entire subgraph is grafted
into the PC's PCGenGraph, it includes objects that the PC is not qualified for.  This means that
a graph traverse must be done at various points to determine if the character is eligible for 
the items present in the PC's PCGenGraph.  This traversal is non-trivial because it is a 
conditional traverse.
}

\newtextPointZeroFour{
If the traversal is done as a top-down (``determine all active nodes'') check, then the 
Prerequisite objects as defined today will fail.  The problem lies in the fact that a graph
traverse is reliant on Prerequisite testing.  If the Prerequisite tests also require
traversal of the graph (which they do, since presence of a PObject in the PC's PCGenGraph
does not mean the character actually has the PObject), then there exists the possibility of 
an infinite loop.  Such an infinite loop is not hard to produce - A Template that grants a Skill 
which is a also present in a \ix{Prerequisite} of the Template will cause this to occur (In other words,
it is common enough that it must be considered).
}

\newtextPointZeroFour{
There are two ways around this: One, make \ix{Prerequisite}s aware of graph traversal (effectively tag
those objects which are known to have been visited and don't visit them twice).  This is considered
to be ugly, because it is providing awareness of the Graph traversal into objects which should
be blissfully unaware of any such activity.\footnote{There are thread safety issues here too, but
that is a future consideration}
}

\newtextPointZeroFour{
Second, \newtextPointTwo{and the proposed solution,}
\ix{Prerequisite}s could be rebuilt to test a PCGenGraph rather than the PlayerCharacter.
In this way, an active PObject SubGraph could be built and the Prerequisite system could be asked
to test against that SubGraph.  This allows a graph traversal to occur in a way that deterministically
traverses the entire graph, and resolves which objects are available and which are not.
}

\newtextPointZeroFour{
If a bottom-up traversal is performed, there is still risk of an infinite loop.  It is postulated 
that the same infinite loop would be produced by the same Template mentioned above.  This would 
also require the \ix{Prerequisite} system be traverse-aware, or that the \ix{Prerequisite} system be rebuilt.
}

\lsubsubsection{Active Status of Negative Prerequisites}

\newtextPointThree{
There is a very specific case in MSRD that highlights a situation that must be carefully
considered in active object resolution in a PlayerCharacter: Negative Prerequisites.
}

\newtextPointThree{
Consider the example in MSRD occupations that either makes a skill a class skill for all
classes of the character, or adds the +1 bonus to the skill if it's already a class skill.
Each Occupation calls a number of hidden feats for skills. If the skill is a class skill
of the base class, then you get a +1 competency bonus to it. If it isn't, it becomes a
class skill. It's something of the form: Ability A awards Ability B (which has
PRECSKILL:1,Foo) or Ability C (which has !PRECSKILL:1,Foo).
}

\newtextPointThree{
Note specifically the !PRECSKILL.  When active object resolution is performed, every
Prerequisite can be evaluated as either true, false, or unresolved.  Any positive 
Prerequisite (e.g. PRECLASS:1,Fighter) may be proven true when viewing a subset of a PC,
but it cannot be proven false until the character is fully resolved.  The opposite is
true of a negative Prerequisite.  However, it is critical to distinguish the two
unresolved cases.  While an unresolved Positive Prerequisite should result in the 
connected objects being ignored, an unresolved negative Prerequisite should 
result in the connected objects being included in the PC.  Therefore, the activation
of these otherwise unresolved negative prerequisites must be performed and the order
of activation carefully considered.
}

\newtextPointThree{
Note also another consequence of this: It is possible to create an object hierarchy
that would cause an infinite loop to an active object resolution system that is
not acutely aware of such a possibility.  This should not be insurmountable, but
requires careful implementation of the active object resolution system.
}

\lsubsection{XP Penalty}

\deletedTextPointThree{\newtextPointZeroTwo {
XP Penalties are interesting.  One opinion would be to have the XP Penalty a 
function of the GameMode and to teach the system to properly calculate the XP
Penalty for any given level increase.
}}

\revisedtextPointThree{
XP Penalty will be made a content object (like SR, DR), attached to the master
PCGenGraph as with any other object.  Any object that wishes to modify an XP
Penalty can create an XPPenalty Modifier to return the appropriate XPPenalty
to be associated with the parent object of the original XPPenalty.  This
penalty could be zero (as in the case of a Preferred Class), or it could be 
some other value (I can't say I know a d20-like system that does this, but 
it's possible!).
}

\deletedTextPointThree{\newtextPointZeroFour{
Actually, there is a much cleaner way to do this: A FavoredClass can be stored
as a primary object in the graph.  It acts as an XPPenalty Suppressor\ldots
It can be queried first, ``hidden'' behind a PRECONTEXT
for the appropriate PCClass, and then the PCGenGraph can
be walked to extract all of the XPPenalty Objects which are not children of the
FavoredClass.
}}

\deletedTextPointThree{\newtextPointZeroFour{
While that may not be entirely clear, also think what could be done to allow
a PObject to suppress the ``Exclusive'' trait of a Skill.  This ``other object 
suppression'' seems to be a behavior used more than once, and that type of
behavior should be standardized in some fashion.
}}


\lsubsection{Movement and Hit Die Locking}

\newtextPointThree{
Given the concept of a Modifier, as discussed above for XP Penalties, 
Modifiers for Movement and Hit Die Locking also become possible (and understandable). 
As a Modifier can be given all of the available objects, this allows the Modifier concept to 
do all of the forms of addition and replacement that are required of Movement.
Hit Die Locking becomes possible with a few different types of Modifiers (one for stepping,
one for locking, etc.).
}

\sbeffect{Modifier Order of Operations}{\newtextPointThree{I think this may actually
require some sort of ranking of Modifiers, so that multiply could occur before add,
and that type of thing for Movement.  Not rocket science, by any means, just required
to properly replicate the movement capabilities of PCGen 5.11.*.  There are more
challenging issues for Hit Die Locking, which are dealt with in Section
\ref{Hit Die Locking Interaction}.
}}

\lsubsection{Retroactive Tokens}

\newtextPointThree{
It is recognized that there are order of operations issues with certain tokens (such 
as hit die locking).  These issues must be resolved so that a CDOM environment can
properly calculate PC characteristics.  Therefore, any token that should be
applied retroactively must be prefixed with \texttt{RETRO:YES:} (or \texttt{RETRO:NO:} 
to suppress retroactive application of a token that is retroactive by default.  
}

\newtextPointThree{
At this time, only tokens related to skills, skill points, and hit dice are
recognized as being potentially retroactive.\footnote{also
see: \url{http://tech.groups.yahoo.com/group/pcgen_experimental/message/7055}
}
}

\lsubsection{Unrelated Information}

\newtextPointZeroFour{
This is a `relative' of the Order of Operations issue.
We need to define how we will handle unrelated information - e.g. if a Race awards a
Class Skill through CSKILL. \revisedtextPointThree{In this case, all classes are allowed to use
the awarded skill as a Class Skill}.
}

\newtextPointThree{
CSKILL and CCSKILL should be global when granted from outside a class. This is used in
the Starting Occupation hidden feats in Modern.  In order to clarify how CSKILL will be 
used and awarded by CDOM, it would be beneficial to have new tokens for class
and cross class skills in Classes.
}

\newtextPointThree{
This new token will ensure that CSKILL and CCSKILL are global when used in any file except
a Class LST file, so even if an Ability is granted by a Class and has
a CSKILL Token, it impacts ALL classes for that PC in the future
}

\newtextPointThree{
We introduce:\\
\texttt{CLASSSKILL:<classname>|<skillname>|<skillname>|PRExxx}\\
\texttt{CROSSCLASSSKILL:<classname>|<skillname>|<skillname>|PRExxx}\\
...with a few hardcoded words to replace <classname>:
\begin{itemize}
\item CURRENT for use on Class / class level lines to mean the given class only.
\item DOMAIN or DOMAINGRANTED for use in domains.lst so that a class(es)
that grants the domain gets the cskill.
\end{itemize}
If we do that, we could turn CSKILL to behave the same way everywhere
(always awards to all classes).
}

\lsubsection{Removal of PObjects}

Example: Template's REMOVE:

In some cases, it is necessary to evaluate existing associations in the PCGenGraph.  This
will require some form of active pruning of the PlayerCharacter PCGenGraph.  The challenge
is in how to STORE the need for that pruning (since the list is dynamic)

\newtextPointThree{
Generally, this is difficult to do without introducing a lot of complexity.  The solution
must consider a number of cases.  The following examples cover attribute locking (this 
was a discussion on \_experimental), but in general, the cases also apply to Template removal.
Given: Template A locks STR, Template B clears the
lock on STR, Template C also locks STR.  Here are the cases we need to consider:
\begin{itemize}
\item Add A, Add B, Remove B (restores A)
\item Add A, Add B, Remove A, Remove B (must not restore the lock from A
- the removal can't hide the lock from being removed if its parent is
removed)
\item Add A, Add C, Add B (B unlocks A)
\item Add A, Add C, Add B, Remove A (B guarantees unlock of C)
\item Add B, Add C (result depends on retroactivity of the lock in C)
\item Add A, Add B, Add C (result depends on retroactivity of the lock in C)
\item Add A, Add B, Add C, Remove A (result depends on retroactivity of
the LOCK in C)
\item Add A, Add B, Add C, Remove object that granted A (result depends on
retroactivity of the LOCK in C)
\item Add A, Add B, lose B due to Prerequisite change (restores A)
\end{itemize}
}

\newtextPointThree{
There are also some cases unique to abilities, such as: Take Multiple:NO Feat A,
Remove A, Choose A, then delete the removal.  This requires the second feat
selection of Feat A to be a rules violation.  This should not be automatically 
cleaned up (guessing which feat slot to empty) as there is an interaction with 
retroactivity that requires user intervention.
}

\newtextPointThree{
The proposed solution here is a removal object slot.  This removal object identifies
the source and sink (the sink being the removed object), and keeps the connecting edge.
This removal object will need to monitor the graph to ensure it remains active, as well 
as to ensure that the grantor of the removed object is not eliminated from the graph.
}

\lsubsection{Hit Die Locking Interaction}

\newtextPointZeroFour{
Currently, in the PCGen 5.11.* series, Hit Die Locking is broken.  This is related
to the order of operations and what to do in indeterminate cases.
}

\newtextPointZeroFour{
As an example of the problem, assume for a Moment that a Class receives a D10 Hit Die.
}

\newtextPointZeroFour{
If we apply a Race that provides a ``Hit Die Lock'' of \%/2 (effectively
dividing the Hit Die by two), then we end up with a D5.  This alone is
not a problem, although this Hit Die size is not defined in the Global 
Hit Die List (which is hardcoded :( ).
}

\newtextPointZeroFour{
However, if a Template then applies \%up1 (which is supposed to step up
one die size), we will end up with a D12.  If we used \%Hup1, we'd end up
with D1000.  I highly doubt this is the intent, but what should we be
doing to catch these situations where Hit Die Locking may cause
unexpected behavior?
}

\newtextPointZeroFour{
It is not quite possible to do this today in RSD35, but only because
making an Animated Object (Tiny) that is also Half Dragon is prohibited
by Alignment.  Therefore, while theoretical in the PCGen \ix{LST file}s,
it might be possible elsewhere and we should consider how it's handled
(in my opinion).
}

\newtextPointZeroFour{
As another case, the issue of how Templates are applied (and the fact
that they may be order dependent with regard to Hit Die Locking) should
also be considered.
}

\newtextPointZeroFour{
This last point cannot necessarily be resolved simply by stating what order
items were added to the character, as two conflicting templates may be 
added by the same PCClassLevel.  Therefore, some more advanced system 
(preferably NOT based on the order of the \ix{LST file}) needs to be adopted.
}

\revisedtextPointThree{
Directly conflicting Hit Die Locks will be prevented, since the PCGenGraph 
will not store an order of operations.  It will be legal to have a retroactive
Hit Die Lock applied at 2nd level and another at 4th level.  However, it will 
be illegal to have two incompatible retroactive hit die locks both applied at
4th level.
}

\lsubsection{Multiple Awarded objects and pools}

\newtextPointThree{
While PCGen 6.0 will move away from one large pool for skills, feats, and other
such objects, it is not possible to completely eliminate pools for these types
of objects.  These are required because it could be possible to have a Feat
which grants two skills.  The Feat cannot simply have a slot for two skills
attached to the Feat: If the Feat is taken a second time, the slot must
expand to four skills.  Therefore, the concept of Slots must be slightly 
expanded to include dynamic Slots (though I would note these are localized
pools, not a global pool).}

\lsubsection{Order of Operations}

\newtextPointZeroThree{
The order of operations and how Skills, Feats, and other slot based items
are associated with PObjects in the PlayerCharacter PCGenGraph needs to
be resolved.
}

\newtextPointZeroThree{
We don't want PCGen as a piece of software to be as constraining as the rules;
otherwise, you 
could never efficiently add more than one level at a time. Therefore,
since Skill and feat slots cannot be saved form one level to another, any
unused slots that are not deployed before leveling up must NOT be impacted
by anything gained at a subsequent level. This will
require a deep association and prerequisite-based search of what the PC was
when the slot was awarded, not what the PC currently is.
\deletedTextPointThree{
\newtextPointZeroFour{
This is heavily dependent on getting the structure of theoretical information
in the PCGenGraphs resolved... given PRECONTEXT, the order of operations can
be reasonably enforced.
}}
}

\newtextPointZeroFour{
The main challenge is what to
do with Templates that are added after Level 1.  Associating them with the 
active PCLevel makes PCLevel `special' (which is perhaps what we want to do),
but also ensures that it will not impact items retroactively.  
\newtextPointThree{
At this point, I think this is the easiest solution, although in the future
a better solution may be developed.
}
}

\deletedTextPointThree{
One alternative
to making PCLevel special is for the PlayerCharacter to maintain an ordered list 
of the immediate children of the Player Character root... and when adding slots
in one of the branches, to suppress all branches that were added to the 
character after the branch with the slot being filled.
}

\deletedTextPointThree{
\newtextPointZeroFour{
With this last item (of an order dependent list) the question is: How is a 
\ix{contract} avoided (since duplicate information is being stored) and how
is that information serialized and restored when a PC is \index{save}saved?
(especially since items are ordered)
}
}

\lsubsection{Followers and Companions}

\newtextPointThree{
This requires either a new form of edge or a new content object.  This is not
terribly complicated to store.  Processing may require special processing
by the classes that extract information from the PC's PCGenGraph.
}

\newtextPointZeroFour{
As a random tangent, however, should we assign each PC a unique-ish identifier 
to be stored in the file that would allow the system to detect changes on the disk 
(file in one location was overwritten by another character??).  This level of 
integrity, while perhaps a bit overkill may be necessary to help debug PC 
differences if, as suggested in section \ref{Saving Characters}, the entire
graph of objects representing the PlayerCharacter is written out.  (I assume here
that the companions are not serialized into the PlayerCharacter, as that would get
really ugly)}

\lsubsection{Connection Restrictions}

\newtextPointThree{
Example: ITEM: in Spell LST
}

\newtextPointThree{
The ITEM Token (as well as a lot of the container/composition proposal information
from equipment) creates source or sink limitations on certain objects.  Therefore,
each object will be capable of defining source or sink restrictions to define the
other objects to which it can be connected.
}

\newtextPointThree{
As a side effect of this proposal, in order to eliminate an order of operations issue,
it is proposed that the primary type of a piece of equipment (Potion, Scroll, Wand, etc.)
be converted to a Category.  In this way, it cannot be modified by an 
EquipmentModifier (using ITYPE).  If it was possible to modify a piece of equipment 
with an ITYPE to modify its primary type, then there is an order of operations 
resolution that would have to take place on Spell addition to an object and 
Equipment Modifiers.\footnote{This is possible, but there is an issue of how often
we want to run certain code and the impact of that.  I think it's best to only check
source and sink restrictions at connection time, and not repeatedly check every
restriction after every object addition.  Such repeated checking becomes exponential
in time, which is ugly}
}

\lsubsection{Overrides}

\newtextPointZeroFour{
There are at least two tags (CHANGEPROF and ARMORTYPE) that override some operating information
in the PObjects to modify behavior.  This override behavior should probably be generalized. 
}

\newtextPointThree{
There are two ways to handle this.  First, would be the use of the Modifier model, but to have
the modifiers capable of modifying before tests occur.  I'm not terribly fond of it, but it
may work well.
}

\newtextPointThree{
The other option here is to have these object act as context-dependent Equipment Modifiers. 
Thus, there would be a Type Modifier attached to an Activator, much like the old ``third party 
token'' system.  This would
then be conditionally applied to the Equipment.  This solution would likely require some form
of order for modifiers (though that thought of `order' may also be applicable elsewhere), in 
case a real EqMod Type Modifier was already present on the Equipment.
}

\lsection{Deferred as non-critical}

\newtextPointTwo{
It is recognized that some changes are nice, but can (and should) be held outside the scope
of the CDOM change.  To attempt to control the scope of the
plans to make sure we only do what is highly beneficial (for efficiency)
or absolutely required, this section is a placeholder for deferred ideas.
}

\lsubsection{Code sharing in Tokens}

\index{token}
\newtextPointZeroTwo {
Currently, the Token system doesn't share much common code between similar (or identical) 
tokens.  Some attempt is made at doing this by placing tokens into the GlobalLSTToken 
interface, and then gating out certain types.\footnote{By
the way, Tom feels this is a bad practice,
\revisedtextPointZeroFour{
the tokens should not be performing instanceof tests - it should be clear from their interface
which PObjects they serve if it is truly not a Global Token.}}
There are some tokens (such as WeaponBonus)
that appear in multiple \ix{LST file} types.  It would be beneficial to share processing between
those Tokens WITHOUT placing the decoding logic into the core.
\revisedtextPointZeroFour{It would also be nice to have a method of not even duplicating 
the Tokens to guarantee zero risk of overloading of the same name in more than one file
(thus limiting the risk of those overloads having different processing of the token).
The trick in making tokens universal is that some things
just shouldn't be legal everywhere, and errors should not be silently consumed.
(A PCClass, for example, shouldn't be able
to define a Face... but the Face should be universally processed between
Race, Template, etc.)
}
}

\newtextPointZeroTwo {
Perhaps this means there needs to be another Interface that causes a direct
import of a Java Class into the current runtime (by using the default ClassLoader)
so that other Classes can directly reference it (rather than the only-indirect 
to an instantiation method for current \index{token}tokens).
\revisedtextPointZeroFour{
Better yet, it would be nice to have a token interface that allows extension by other tokens.
}
}

\newtextPointZeroFour{
Another alternative here is to have a \index{token}Token system that specifically defines what Classes it 
can impact.  For example:\\
\texttt{\noindent public interface PToken<T extends PObject> extends LstToken \{\\
\indent public boolean parse(\ix{LoadContext} lc, T cdomObj, String s);\\
\indent public List<Class<T>> getHandledClass();\\
\noindent \}}\\
}

\newtextPointZeroFour{
This allows two benefits, without significant added complexity to the \index{token}Token system.  First, 
allowing tokens to extend other tokens would not be required, as identical tokens in different
PObjects could be handled simply by the getHandledClass method.  Also, since these `generic' 
Tokens can be stored in the TokenStore by the actual PObject that they modify (rather than
by the LstToken interface that they implement), this would allow for a `GenericLoader' to be
created, which would significantly cut down the number of Classes required to implement the 
loading system (PCClass would still require special handling, due to the nature of PCClassLevels,
and Kits would require special handling due to the nature of their layout, but the rest of the
basic PObjects could be handled by the GenericLoader).
}

\newtextPointZeroFour{
\noindent Since this gets a little confusing for single vs. multiple PObject Tokens, here is a single 
PObject \index{token}token: \\
\texttt{\noindent public interface SkillSomethingToken extends PToken<Skill> \{ \\
\indent public boolean parse(\ix{LoadContext} lc, Skill aSkill, String s) \{ \\
\indent ~~return true; \\
\indent \} \\
\indent public List<Class<T>> getHandledClass() \{ \\
\indent ~~return Collections.singletonList(Skill.class); \\
\indent \} \\
\noindent \} } \\
}

\newtextPointZeroFour{
\noindent Here is a multiple PObject \index{token}token: \\
\texttt{\noindent public interface SPropToken extends PToken<PObject> \{ \\
\indent public boolean parse(\ix{LoadContext} lc, T cdomObj, String s) \{ \\
\indent ~~return true; \\
\indent \} \\
\indent public List<Class<T>> getHandledClass() \{ \\
\indent ~~List<Class<T>> list = new ArrayList<Class<T>>(); \\
\indent ~~list.add(Equipment.class); \\
\indent ~~list.add(EquipmentModifier.class); \\
\indent ~~return list; \\
\indent \} \\
\noindent \} } \\
}

\newtextPointZeroFour{
The direct import would still be valuable.  For example, it would be
beneficial to have processing for CHOOSE or ADD to be sent out to a SubLoader\index{loader} for 
processing.  However, placing the SubLoader back into the Core doesn't seem appropriate.
That SubLoader should be able to be placed into the \index{plugins}\texttt{plugin.*}
directories and somehow tagged
so that it can be directly referenced by the \index{token}Tokens.  This is perhaps best done in the
Manifest File for the particular Plugins(?).  This functionality would then provide
the ability to create complex tokens without having access to the core of PCGen.
}

\newtextPointTwo{
Actually, what I would really like to see here is the ability to have a Manifest file
drive load of the classes.  I think the ChooseLoader and AddLoader (as well as many of the 
Kit*Loaders, etc.) be placed out in the \texttt{plugin.*} directory structure and not require
any code in the core to do complex tokens.
}

\lsubsection{PRExxx Token consistency and support}

\newtextPointTwo{
This encompasses things like the recent CSKILL doesn't
support prereqs discussion. There is also significant inconsistency between 
different tokens.  Currently you can define the
prereq using TAG:name\vbar PREREQ or TAG:name[PREREQ] or not at all,
depending on the tag.
} 
 
\rejected{

\lsection{Rejected Proposals}

\lsubsection{Unpassed Prerequisites}

Replaced in preference of directly importing all objects into the PC Graph and doing a complete
Graph search when information is required.

As already noted, there are PObjects that possess \ix{Prerequisite}s and where the associations to
other PObjects contain \ix{Prerequisite}s in the connecting edges.  In order to increase the speed
of processing for a `static' PlayerCharacter it is proposed that only objects which the
PlayerCharacter qualifies for are placed into the PlayerCharacter's PCGenGraph.  This allows
for very efficient processing (grab what is in the PlayerCharacter's PCGenGraph and GO\ldots no
\ix{Prerequisite} testing is required).  When a Prerequisite for an additional object is not fulfilled,
a PrerequisiteLink (a form of hyperedge with one source and no sink) can be placed into the
PlayerCharacter's PCGenGraph.  These links can then be tested when a PlayerCharacter is updated,
and if they are subsequently passed, the PrerequisiteLink can contain enough information to
import the appropriate nodes and edges from the master PCGenGraph into the PlayerCharacter's
PCGenGraph.

\lsubsection{Factories for Insertion}

Replaced by Object Addition on Other Object Insertion (section
\ref{Object Addition on Other Object Insertion}).

\revisedtextPointZeroThree{
This is a similar, but slightly different case than \ref{Object Addition on Self Insertion}.
While that was based on the object
being inserted into a PlayerCharacter, Case \ref{Object Addition on Other Object Insertion} 
requires that objects within the PlayerCharacter
be queried or notified of changes to the PlayerCharacter.  (push vs. pull to be determined). 
This is required because those objects may wish to add additional objects to the PlayerCharacter
if a given PCLevel is reached, for example.
}

\revisedtextPointZeroThree{
Case \ref{Object Addition on Other Object Insertion} brings
up a very interesting point.  Assume that a Template (e.g. ``Winged'') adds a Feat
(e.g. ``Elite Jump'') to a PlayerCharacter when that PC achieves character level 9.  This could
be added to the PlayerCharacter when the Template is added, or even just statically linked to
the Template with the addition of a \ix{Prerequisite} on the edge based on the PCLevel.  However,
I believe it is superior to actually represent this association between the PCLevel, the Template
and the Feat in the PCGenGraph.  Therefore, I would use the Factory method from use cases
\ref{Object Addition on Self Insertion} and \ref{Object Addition on Other Object Insertion},
and propose that edges within the PCGenGraph are not limited to only having a single source and a
single sink.  It should be possible for an edge to have more than one source.  Thus, the Feat
would actually be sourced from BOTH the PCLevel(9) and Template.  This gives a visual connection
that the Feat is a function of BOTH the Template and the PCLevel(9).  Given the rules of how
such a hyperedge would be removed from the graph (it is removed if any single source or sink
was removed), the Feat would be lost if either the Template or PCLevel was lost by the
PlayerCharacter.
}

\lsection{Temporary Unqualified Objects with Optimized PC}

\revisedtextPointZeroFour{
In some cases, a PlayerCharacter will make a choice (of a Feat, for example) that may have
prerequisites that the PlayerCharacter later becomes ineligible for.  Because this choice
information is unique to the PlayerCharacter, the example of a PrerequisiteLink from Case
\ref{Unpassed Prerequisites} (above)
is insufficient to maintain the choice that the PlayerCharacter has made. 
Therefore, an alternative method of storing that choice (perhaps by allowing the
PrerequisiteLink to contain a small PCGenGraph) while keeping it out of the PlayerCharacter's
PCGenGraph is required.
}

}

\lnsection{Packages}

\lsubsection{Proposed Packages}

\noindent\texttt{pcgen.base.*}

These are the \ix{library} components of pcgen.  These Classes and methods are ignorant of PCGen
and CDOM, and only refer to the core java.* Classes provided in the JDK (though exceptions
for some critical jars we import can be discussed, I don't think any are sufficiently general
enough for the interface to them to be treated as a library element)

These Classes, as library elements, should be strictly enforced to be 
\revisedtextPointTwo{internationalized,} 100\% documented with
useful comments, their semantic behavior, and have 100\% \ix{unit test} code coverage. *With the
understanding that Emma's \ix{unit test} code coverage will never reach 100\% if you ever use
`Something.class' (due to how it explodes to multiple statements in the code) and we should
never have to test to get to an UnreachableError (see the Corollary 
to \ref{Minimize Error Suppression and Gating}).

\noindent\texttt{pcgen.base.graph.*}

Specifically, this contains the graph components that make up the framework of CDOM. 
Nothing in here is PCGen aware, so it is entirely composed of Generic Classes that make
up a graph.  Same \revisedtextPointTwo{internationalized,} 100\% doc, semantic, and testing
rules apply here.

\noindent\texttt{pcgen.cdom.*}

More detail below, but note that these Classes are only permitted to reference each other
and the libraries (java.*, javax.*, pcbase.*, the jars we import) 

\noindent\texttt{pcgen.cdom.base}

Base Classes used to build a CDOM environment (these are constants, and the non-Generic
version of the pcbase.graph.* Classes).  This package might be merged with pcgen.cdom.core,
depending on the final size and relationships of the Classes.

\noindent\texttt{pcgen.cdom.content}

Content composed of a separate object that may be associated with a PObject (such as
DamageReduction)\ldots will appear independently in the PCGenGraph.

\noindent\texttt{pcgen.cdom.core}

The core parts of CDOM, specifically the abstract objects that are the super Classes 
(PrereqObject, PObject or its successor)

\noindent\texttt{pcgen.cdom.enumeration}

\index{enumeration}
This package contains \ix{type safe} enumerations used in CDOM.  Will be a mix of Java 1.4
and 1.5 typesafe enumerations (some are dynamically expandable)

\noindent\texttt{pcgen.cdom.helper}

Helper Classes to the PObjects (e.g. SpellSupport)

\noindent\texttt{pcgen.cdom.inst}

The actual primary elements of CDOM (PCClass, Language, Skill)

\noindent\texttt{pcgen.cdom.prereq}

Some core interfaces for \ix{Prerequisite}s.

\noindent\texttt{pcgen.gui.*}

(The \ix{GUI} has not yet been properly evaluated)

\noindent\texttt{pcgen.io.*}

Generally unchanged, with some items moved into \texttt{pcbase.*}

\noindent\texttt{pcgen.persistence.*}

\index{Persistence}
More detail below, but the Classes are able to reference each other as well as anything
\texttt{pcgen.cdom.*} Classes may reference.

\noindent\texttt{pcgen.persistence.loader}

The \ix{loader}s\ldots

\noindent\texttt{pcgen.persistence.prereq}

The \ix{Prerequisite} Classes.

\noindent\texttt{pcgen.persistence.token}

\index{token}Support Classes for the Tokens, specifically the Interfaces that the tokens implement
(PCClassLstToken)

\noindent\texttt{pcgen.persistence.util}

Utility Classes for the persistence system.

\noindent\texttt{pcgen.util}

Much may move to \texttt{pcbase.*}, though some will remain.

\noindent\texttt{pcgen.plugin.*}

\index{plugins}Moved from plugin.*, but generally unchanged, although I
would advise that the \texttt{lsttoken.gamemode.*} and
\texttt{lsttokens.statsandchecks.*}
\newtextPointZeroTwo {
(and those related to the GameMode)
get moved into \texttt{plugin.modetokens.*}
}
in order to clarify
that they are the gamemode lsts, not the data lsts.
\newtextPointZeroTwo {
Similarly, the sponsor plugin should be moved into plugin.sponsor; it should
not be part of \texttt{plugin.lsttokens.*} or \texttt{plugin.modetokens.*}
}
 
\noindent\texttt{gmgen.*}

Need some evaluation here.  Much of this can remain, but there are core references out
to gmgen, and gmgen should be a parallel to \texttt{plugin.*} not a core set of Classes. 
Some of this should be moved back into core.

\lnsubsection{Package Relationships}

\mapfig{packages.png}{\label{Fig: Package Relationships}PCGen CDOM Package Relationships\index{export}\index{GUI}}

Figure \ref{Fig: Package Relationships} shows the relationships between the packages (as proposed).
Items may only call to packages below them - no calling up the stack.  The \ix{GUI} is considered a form
of export, for purposes of this Figure.

\newpage
\printindex

\end{document}

